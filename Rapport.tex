%%%%%%%%%%%%%%%%%%%%%%%%%%%%%%%%%%%%%%%%%%%%%%%%%%%%%%%%%%%%%%%%%%%%%%
%%%%% Configuration %%%%%%%%%%%%%%%%%%%%%%%%%%%%%%%%%%%%%%%%%%%%%%%%%%
%%%%%%%%%%%%%%%%%%%%%%%%%%%%%%%%%%%%%%%%%%%%%%%%%%%%%%%%%%%%%%%%%%%%%%

\documentclass[
	a4paper,	% feuille A4
	twoside,	% impression Recto-Verso
	11 pt		% taille de police
]{report}		% type de document


%%%%%%%%%% Extensions %%%%%%%%%%
\usepackage[utf8]{inputenc}		% encodage des caractères
\usepackage[T1]{fontenc}		% nouvelle norme de codage
\usepackage[francais]{babel}	% langue française
%\usepackage{lscape}				% format paysage
\usepackage{multicol}			% multi-colonnes
\usepackage{graphicx}			% images
%\usepackage{float}				% insertion d'image avec l'option "H"
%\usepackage{textcomp}			% signe €
%\usepackage{ulem}				% souligner


%%%%%%%%%% Marges %%%%%%%%%%
\usepackage[
	tmargin = 2.5 cm,	% haut
	bmargin = 5 cm,		% bas
	lmargin = 2.5 cm,	% gauche
	rmargin = 2.5 cm	% droite
]{geometry}


%%%%%%%%%% Inter-ligne %%%%%%%%%%
\usepackage{setspace}
\onehalfspacing			% 1.5


%%%%%%%%%% Code source %%%%%%%%%%
%\usepackage{listings}
%\lstset{
%	showstringspaces = false,	% masquer les espaces
%	tabsize = 4,				% nombre d'espace par tabulation
%	basicstyle = \ttfamily,		% caractères de même taille
%	xleftmargin = 30 pt,		% marge à gauche
%	xrightmargin = 0 pt,		% marge à droite
%	aboveskip = -1 pt,			% espace au dessus de la zone
%	belowskip = -1 pt			% espace en dessous de la zone
%}


%%%%%%%%%% Informations %%%%%%%%%%
\title{Stage à Sopra Group}
\author{\href{mailto:pinguet62@gmail.com}{Julien PINGUET}}
\date{Avril-Septembre 2013}


%%%%%%%%%% Liens %%%%%%%%%%
\usepackage{hyperref}
\hypersetup{ 			% Options : http://forum.ubuntu-fr.org/viewtopic.php?id=81505%29
	backref = true,
	pagebackref = true,	% dans bibliographie
	pdfborder = {0 0 0}	% retirer le cadre
}


%%%%%%%%%% Glossaire %%%%%%%%%%
%\usepackage{glossaries}
%\makeglossaries
%\cleardoublepage

\chapter*{Glossaire}

\addcontentsline{toc}{chapter}{Glossaire}
\markboth{Glossaire}{}



TODO: commandes LaTeX

Bibliothèque : TODO

Framework : ensemble de composants logiciels permettant le développement le développement rapide d'applications.


%%%%%%%%%% En-tête et Pied de page %%%%%%%%%%
\usepackage{fancyhdr}
\makeatletter % début d'utilisation des variables @



%%%%%%%%%%%%%%%%%%%%%%%%%%%%%%%%%%%%%%%%%%%%%%%%%%%%%%%%%%%%%%%%%%%%%%%%%%%%%%%%%%%%%%%%%%%%%%%%%%%%
%%%%% Page vides %%%%%%%%%%%%%%%%%%%%%%%%%%%%%%%%%%%%%%%%%%%%%%%%%%%%%%%%%%%%%%%%%%%%%%%%%%%%%%%%%%%
%%%%%%%%%%%%%%%%%%%%%%%%%%%%%%%%%%%%%%%%%%%%%%%%%%%%%%%%%%%%%%%%%%%%%%%%%%%%%%%%%%%%%%%%%%%%%%%%%%%%

\fancypagestyle{empty}{
	% Effacer les valeurs par défaut
	\fancyhf{}
	
	% Remplissage
	\lhead	[]								% haut	gauche	pair
			{}								% haut	gauche	impair
	\chead	[]								% haut	centre	pair
			{}								% haut	centre	impair
	\rhead	[]								% haut	droite	pair
			{}								% haut	droite	impair
	\lfoot	[]								% bas	gauche	pair
			{}								% bas	gauche	impair
	\cfoot	[]								% bas	centre	pair
			{}								% bas	centre	impair
	\rfoot	[]								% bas	droite	pair
			{}								% bas	droite	impair
	
	% Epaisseur de la ligne séparatrice
	\renewcommand{\headrulewidth}{0 pt}		% en-tête (défaut : 0.4pt)
	\renewcommand{\footrulewidth}{0 pt}		% pied de page (défaut : 0pt)
	
	% Distance du corps
	\headsep = 25 pt						% en-tête (défaut : 25pt)
	\footskip = 75 pt						% pied de page (défaut : 30pt)
}



%%%%%%%%%%%%%%%%%%%%%%%%%%%%%%%%%%%%%%%%%%%%%%%%%%%%%%%%%%%%%%%%%%%%%%%%%%%%%%%%%%%%%%%%%%%%%%%%%%%%
%%%%% Page des chapitre (défaut) %%%%%%%%%%%%%%%%%%%%%%%%%%%%%%%%%%%%%%%%%%%%%%%%%%%%%%%%%%%%%%%%%%
%%%%%%%%%%%%%%%%%%%%%%%%%%%%%%%%%%%%%%%%%%%%%%%%%%%%%%%%%%%%%%%%%%%%%%%%%%%%%%%%%%%%%%%%%%%%%%%%%%%%

\fancypagestyle{plain}{
	% Effacer les valeurs par défaut
	\fancyhf{}
	
	% Remplissage
	\lhead	[\large \bf \fbox{\thepage}]	% haut	gauche	pair	:	numéro de page
			{}								% haut	gauche	impair
	\chead	[]								% haut	centre	pair
			{}								% haut	centre	impair
	\rhead	[]								% haut	droite	pair
			{\large \bf \fbox{\thepage}}	% haut	droite	impair	:	numéro de page
	\lfoot	[\@title]						% bas	gauche	pair	:	titre du rapport
			{\@author}						% bas	gauche	impair	:	auteur du rapport
	\cfoot	[]								% bas	centre	pair
			{}								% bas	centre	impair
	\rfoot	[\@author]						% bas	droite	pair	:	auteur du rapport
			{\@title}						% bas	droite	impair	:	titre du rapport
	
	% Epaisseur de la ligne séparatrice
	\renewcommand{\headrulewidth}{0 pt}		% en-tête (défaut : 0.4pt)
	\renewcommand{\footrulewidth}{0.4 pt}	% pied de page (défaut : 0pt)
	
	% Distance du corps
	\headsep = 25 pt						% en-tête (défaut : 25pt)
	\footskip = 75 pt						% pied de page (défaut : 30pt)
}



%%%%%%%%%%%%%%%%%%%%%%%%%%%%%%%%%%%%%%%%%%%%%%%%%%%%%%%%%%%%%%%%%%%%%%%%%%%%%%%%%%%%%%%%%%%%%%%%%%%%
%%%%% Corps du rapport (hors page de chapitre) %%%%%%%%%%%%%%%%%%%%%%%%%%%%%%%%%%%%%%%%%%%%%%%%%%%%%
%%%%%%%%%%%%%%%%%%%%%%%%%%%%%%%%%%%%%%%%%%%%%%%%%%%%%%%%%%%%%%%%%%%%%%%%%%%%%%%%%%%%%%%%%%%%%%%%%%%%

\fancypagestyle{corps}{
	% Effacer les valeurs par défaut
	\fancyhf{}
	
	% Remplissage
	\lhead	[\large \bf \thepage]				% haut	gauche	pair	:	numéro de page
			{\bf \nouppercase \rightmark}		% haut	gauche	impair	:	nom de la section
	\chead	[]									% haut	centre	pair
			{}									% haut	centre	impair
	\rhead	[\large \bf \nouppercase \leftmark]	% haut	droite	pair	:	nom du chapitre
			{\large \bf \thepage}				% haut	droite	impair	:	numéro de page
	\lfoot	[\@title]							% bas	gauche	pair	:	titre du rapport
			{\@author}							% bas	gauche	impair	:	auteur du rapport
	\cfoot	[]									% bas	centre	pair
			{}									% bas	centre	impair
	\rfoot	[\@author]							% bas	droite	pair	:	auteur du rapport
			{\@title}							% bas	droite	impair	:	titre du rapport
	
	% Epaisseur de la ligne séparatrice
	\renewcommand{\headrulewidth}{0.4 pt}		% en-tête (défaut : 0.4pt)
	\renewcommand{\footrulewidth}{0.4 pt}		% pied de page (défaut : 0pt)
	
	% Distance du corps
	\headsep = 25 pt							% en-tête (défaut : 25pt)
	\footskip = 75 pt							% pied de page (défaut : 30pt)
}



%%%%%%%%%%%%%%%%%%%%%%%%%%%%%%%%%%%%%%%%%%%%%%%%%%%%%%%%%%%%%%%%%%%%%%%%%%%%%%%%%%%%%%%%%%%%%%%%%%%%
%%%%% Annexes (hors page de chapitre) %%%%%%%%%%%%%%%%%%%%%%%%%%%%%%%%%%%%%%%%%%%%%%%%%%%%%%%%%%%%%%
%%%%%%%%%%%%%%%%%%%%%%%%%%%%%%%%%%%%%%%%%%%%%%%%%%%%%%%%%%%%%%%%%%%%%%%%%%%%%%%%%%%%%%%%%%%%%%%%%%%%

\fancypagestyle{annexe}{
	% Effacer les valeurs par défaut
	\fancyhf{}
	
	% Remplissage
	\lhead	[\large \bf \thepage]			% haut	gauche	pair	:	numéro de page
			{\bf \nouppercase \rightmark}	% haut	gauche	impair	:	nom de la section
	\chead	[]								% haut	centre	pair
			{}								% haut	centre	impair
	\rhead	[\large \bf Annexes]			% haut	droite	pair	:	"Annexes"
			{\large \bf \thepage}			% haut	droite	impair	:	numéro de page
	\lfoot	[\@title]						% bas	gauche	pair	:	titre du rapport
			{\@author}						% bas	gauche	impair	:	auteur du rapport
	\cfoot	[]								% bas	centre	pair
			{}								% bas	centre	impair
	\rfoot	[\@author]						% bas	droite	pair	:	auteur du rapport
			{\@title}						% bas	droite	impair	:	titre du rapport
	
	% Epaisseur de la ligne séparatrice
	\renewcommand{\headrulewidth}{0.4 pt}	% en-tête (défaut : 0.4pt)
	\renewcommand{\footrulewidth}{0.4 pt}	% pied de page (défaut : 0pt)
	
	% Distance du corps
	\headsep = 25 pt						% en-tête (défaut : 25pt)
	\footskip = 75 pt						% pied de page (défaut : 30pt)
}


\makeatother % fin d'utilisation des variables @
% Pas d'en-tête dans la table des matières
\makeatletter
\renewcommand{\tableofcontents}{
	\chapter*{\contentsname}
	\thispagestyle{empty}
	\@starttoc{toc}}
\makeatother


%%%%%%%%%% Notes de bas de page %%%%%%%%%%
\usepackage[bottom]{footmisc}	% Collé au bas de page


%%%%%%%%%% Numérotation des parties %%%%%%%%%%
\setcounter{secnumdepth}{3}		% corps
\setcounter{tocdepth}{3}		% table des matières


%%%%%%%%%% Espacement des parties %%%%%%%%%%
\usepackage{titlesec}
\titlespacing*{\chapter}
	{0pt}							% retrait à gauche
	{0pt}							% espace avant
	{10pt}							% espace après
\titlespacing*{\section}
	{0pt}							% retrait à gauche
	{40pt plus 10pt minus 10pt}		% espace avant
	{10pt}							% espace après
\titlespacing*{\subsection}
	{0pt}							% retrait à gauche
	{30pt plus 10pt minus 10pt}		% espace avant
	{10pt}							% espace après
\titlespacing*{\subsubsection}
	{0pt}							% retrait à gauche
	{20pt plus 10pt minus 10pt}		% espace avant
	{10pt}							% espace après


%%%%%%%%%% En-tête de chapitre %%%%%%%%%%
% Numérotation des chapitres en chiffres romains
\renewcommand{\thechapter}{\Roman{chapter}}

% Format des titres de chapitre
\makeatletter							% début d'utilisation des variables @
\def\@makechapterhead#1{
	\vspace*{50\p@} {					% espace vertical incompressible
		\parindent \z@					% indentation
		\Huge							% grande taille
		\bfseries						% texte gras
		% Si le chapitre porte un numéro
		\ifnum \m@ne < \c@secnumdepth	% secnumdepth : profondeur dans la table des matières
			\thechapter					% numéro de chapitre
			\quad						% espace horizontal moyen
		\fi
		#1								% argument : nom de chapitre
		\par
		\nobreak						% empécher le saut de page/ligne
	}
	\vskip 40\p@						% espace vertical 
}
\makeatother							% fin d'utilisation des variables @

% Format des titres de paragraphe
\newcommand\Jparagraph[1]{	%
	\paragraph{#1}			%
	~\par					% retour à la ligne + alinéa
}


%%%%%%%%%%%%%%%%%%%%%%%%%%%%%%%%%%%%%%%%%%%%%%%%%%%%%%%%%%%%%%%%%%%%%%
%%%%% Contenu %%%%%%%%%%%%%%%%%%%%%%%%%%%%%%%%%%%%%%%%%%%%%%%%%%%%%%%%
%%%%%%%%%%%%%%%%%%%%%%%%%%%%%%%%%%%%%%%%%%%%%%%%%%%%%%%%%%%%%%%%%%%%%%

\begin{document}

\sloppy	% ajuster correctement le texte
\renewcommand{\chaptermark}[1]{\markboth{\thechapter.\ #1}{}}	% titre chapitre
\renewcommand{\sectionmark}[1]{\markright{\thesection.\ #1}}	% titre section
\pagestyle{empty}

%%%%%%%%%% Page de garde %%%%%%%%%%
\thispagestyle{empty}

\makeatletter % début d'utilisation des variables @



\begin{multicols}{2}
	\begin{flushleft}
	
		%%%%%%%%%%%%%%%%%%%%%%%%%%%%%%%%%%%%%%%%%%%%%%%%%%%%%%%%%%%%%%%%%%%%%%
		%%%%% Haut-Gauche %%%%%%%%%%%%%%%%%%%%%%%%%%%%%%%%%%%%%%%%%%%%%%%%%%%%
		%%%%%%%%%%%%%%%%%%%%%%%%%%%%%%%%%%%%%%%%%%%%%%%%%%%%%%%%%%%%%%%%%%%%%%
	
		% Logo école
		\includegraphics[scale=0.09]{img/ISIMA_logo.png}					\\
		
		% Nom école
		\textbf{I}nstitut \textbf{S}upérieur								\\
		d'\textbf{I}nformatique, de											\\
		\textbf{M}odélisation et de											\\
		leurs \textbf{A}pplications	
		
		\vspace*{0.5cm}
		
		% Adresse école
		BP 10125															\\
		63173 Aubière Cedex
		
		%%%%%%%%%%%%%%%%%%%%%%%%%%%%%%%%%%%%%%%%%%%%%%%%%%%%%%%%%%%%%%%%%%%%%%
		
	\end{flushleft}
\columnbreak
	\begin{flushright}
	
		%%%%%%%%%%%%%%%%%%%%%%%%%%%%%%%%%%%%%%%%%%%%%%%%%%%%%%%%%%%%%%%%%%%%%%
		%%%%% Haut-Droite %%%%%%%%%%%%%%%%%%%%%%%%%%%%%%%%%%%%%%%%%%%%%%%%%%%%
		%%%%%%%%%%%%%%%%%%%%%%%%%%%%%%%%%%%%%%%%%%%%%%%%%%%%%%%%%%%%%%%%%%%%%%
		
		%%%%%%%%%%%%%%%%%%%%%%%%%%%%%%%%%%%%%%%%%%%%%%%%%%%%%%%%%%%%%%%%%%%%%%
		
	\end{flushright}
\end{multicols}

\vspace*{\fill}

\begin{center}

	%%%%%%%%%%%%%%%%%%%%%%%%%%%%%%%%%%%%%%%%%%%%%%%%%%%%%%%%%%%%%%%%%%%%%%
	%%%%% Centre %%%%%%%%%%%%%%%%%%%%%%%%%%%%%%%%%%%%%%%%%%%%%%%%%%%%%%%%%
	%%%%%%%%%%%%%%%%%%%%%%%%%%%%%%%%%%%%%%%%%%%%%%%%%%%%%%%%%%%%%%%%%%%%%%

	% Informations
	\Large
	Rapport d'ingénieur													\\
	Projet de 3\up{ème} année											\\
	\textit{Filière :} Génie Logiciel et Systèmes Informatiques			\\
	\textit{Filière :} Réseaux et Télécommunications
	
	\rule{16cm}{2pt}													\\
	\vspace*{0.35cm}
	
	% Titre du projet
	\huge
	\textbf{\@title}													\\

	\rule{16cm}{2pt}
	
	%%%%%%%%%%%%%%%%%%%%%%%%%%%%%%%%%%%%%%%%%%%%%%%%%%%%%%%%%%%%%%%%%%%%%%

\end{center}
	
\vspace*{\fill}

\begin{multicols}{2}
	\vspace*{\fill}
	% Bas-Gauche : Auteurs + Encadreur
	\begin{flushleft}
	
		%%%%%%%%%%%%%%%%%%%%%%%%%%%%%%%%%%%%%%%%%%%%%%%%%%%%%%%%%%%%%%%%%%%%%%
		%%%%% Bas-Gauche %%%%%%%%%%%%%%%%%%%%%%%%%%%%%%%%%%%%%%%%%%%%%%%%%%%%%
		%%%%%%%%%%%%%%%%%%%%%%%%%%%%%%%%%%%%%%%%%%%%%%%%%%%%%%%%%%%%%%%%%%%%%%
	
		% Auteur & Encadrants
		\begin{minipage}{0.7\textwidth} % à adapter en fonction du contenu
			\textit{Présenté par :} \@author
		\end{minipage}
		
		\mbox{\textit{Sous la direction de :} Loïc YON}
		
		%%%%%%%%%%%%%%%%%%%%%%%%%%%%%%%%%%%%%%%%%%%%%%%%%%%%%%%%%%%%%%%%%%%%%%
		
	\end{flushleft}
\columnbreak
	\vspace*{\fill}
	
	\begin{flushright}
	
		%%%%%%%%%%%%%%%%%%%%%%%%%%%%%%%%%%%%%%%%%%%%%%%%%%%%%%%%%%%%%%%%%%%%%%
		%%%%% Bas-Droite %%%%%%%%%%%%%%%%%%%%%%%%%%%%%%%%%%%%%%%%%%%%%%%%%%%%%
		%%%%%%%%%%%%%%%%%%%%%%%%%%%%%%%%%%%%%%%%%%%%%%%%%%%%%%%%%%%%%%%%%%%%%%
		
		% Date
		\@date
		
		%%%%%%%%%%%%%%%%%%%%%%%%%%%%%%%%%%%%%%%%%%%%%%%%%%%%%%%%%%%%%%%%%%%%%%
		
	\end{flushright}
\end{multicols}



\makeatother % fin d'utilisation des variables @


%%%%%%%%%% Remerciements %%%%%%%%%%
\cleardoublepage

\chapter*{Remerciements}

\addcontentsline{toc}{chapter}{Remerciements}
\markboth{Remerciements}{}


Je tiens tout d'abord à remercier les personnes qui m'ont accueilli et accompagnés durant ces 6 mois de stage chez Sopra Group.
\\


Je remercie tout d'abord Delphine JARIGE, chef de projet du projet \textit{Sides}, ainsi que les autres membres de l'équipe, pour mon accueil en ce début de stage.

J'adresse aussi mes remerciements à Laurent LAFFERRERE??, chef de projet du projet \textit{LimaGest}, ainsi que l'architecte Xavier CLEMENCE pour son aide apporter lors des difficultés rencontrées.

Enfin, je remercie l'équipe du projet de l'\textit{??} : Pierre ?? le chef de projet, Ambroise ?? l'architecte, mais aussi les développeurs pour l'aide apporté lors de mon affectation au projet.

%%%%%%%%%% Glossaire %%%%%%%%%%
%\printglossary

%%%%%%%%%% Résumé %%%%%%%%%%
\cleardoublepage

\chapter*{Résumé}

\addcontentsline{toc}{chapter}{Résumé}
\markboth{Résumé}{}



bla bla bla bla bla bla bla bla bla bla bla bla bla bla bla bla bla bla bla bla bla bla bla bla bla bla bla bla bla bla bla bla bla bla bla bla bla bla bla bla bla bla bla bla bla bla bla bla bla bla bla bla bla bla bla bla bla bla bla bla bla bla bla bla bla bla bla bla bla bla bla bla bla bla bla bla bla bla bla bla bla bla bla bla bla bla bla bla bla bla bla bla.

bla bla bla bla bla bla bla bla bla bla bla bla bla bla bla bla bla bla bla bla bla bla bla bla bla bla bla bla bla bla bla bla bla bla bla bla bla bla bla bla bla bla bla bla bla bla bla bla bla bla bla bla bla bla bla bla bla bla bla bla bla bla bla bla bla bla bla bla bla bla bla bla bla bla bla bla bla bla bla bla bla bla bla bla bla bla bla bla bla bla bla bla.
\\


bla bla bla bla bla bla bla bla bla bla bla bla bla bla bla bla bla bla bla bla bla bla bla bla bla bla bla bla bla bla bla bla bla bla bla bla bla bla bla bla bla bla bla bla bla bla bla bla bla bla bla bla bla bla bla bla bla bla bla bla bla bla bla bla bla bla bla bla bla bla bla bla bla bla bla bla bla bla bla bla bla bla bla bla bla bla bla bla bla bla bla bla.

bla bla bla bla bla bla bla bla bla bla bla bla bla bla bla bla bla bla bla bla bla bla bla bla bla bla bla bla bla bla bla bla bla bla bla bla bla bla bla bla bla bla bla bla bla bla bla bla bla bla bla bla bla bla bla bla bla bla bla bla bla bla bla bla bla bla bla bla bla bla bla bla bla bla bla bla bla bla bla bla bla bla bla bla bla bla bla bla bla bla bla bla.
\\


bla bla bla bla bla bla bla bla bla bla bla bla bla bla bla bla bla bla bla bla bla bla bla bla bla bla bla bla bla bla bla bla bla bla bla bla bla bla bla bla bla bla bla bla bla bla bla bla bla bla bla bla bla bla bla bla bla bla bla bla bla bla bla bla bla bla bla bla bla bla bla bla bla bla bla bla bla bla bla bla bla bla bla bla bla bla bla bla bla bla bla bla.

bla bla bla bla bla bla bla bla bla bla bla bla bla bla bla bla bla bla bla bla bla bla bla bla bla bla bla bla bla bla bla bla bla bla bla bla bla bla bla bla bla bla bla bla bla bla bla bla bla bla bla bla bla bla bla bla bla bla bla bla bla bla bla bla bla bla bla bla bla bla bla bla bla bla bla bla bla bla bla bla bla bla bla bla bla bla bla bla bla bla bla bla.
\\


bla bla bla bla bla bla bla bla bla bla bla bla bla bla bla bla bla bla bla bla bla bla bla bla bla bla bla bla bla bla bla bla bla bla bla bla bla bla bla bla bla bla bla bla bla bla bla bla bla bla bla bla bla bla bla bla bla bla bla bla bla bla bla bla bla bla bla bla bla bla bla bla bla bla bla bla bla bla bla bla bla bla bla bla bla bla bla bla bla bla bla bla.

bla bla bla bla bla bla bla bla bla bla bla bla bla bla bla bla bla bla bla bla bla bla bla bla bla bla bla bla bla bla bla bla bla bla bla bla bla bla bla bla bla bla bla bla bla bla bla bla bla bla bla bla bla bla bla bla bla bla bla bla bla bla bla bla bla bla bla bla bla bla bla bla bla bla bla bla bla bla bla bla bla bla bla bla bla bla bla bla bla bla bla bla.
\\


bla bla bla bla bla bla bla bla bla bla bla bla bla bla bla bla bla bla bla bla bla bla bla bla bla bla bla bla bla bla bla bla bla bla bla bla bla bla bla bla bla bla bla bla bla bla bla bla bla bla bla bla bla bla bla bla bla bla bla bla bla bla bla bla bla bla bla bla bla bla bla bla bla bla bla bla bla bla bla bla bla bla bla bla bla bla bla bla bla bla bla bla.

bla bla bla bla bla bla bla bla bla bla bla bla bla bla bla bla bla bla bla bla bla bla bla bla bla bla bla bla bla bla bla bla bla bla bla bla bla bla bla bla bla bla bla bla bla bla bla bla bla bla bla bla bla bla bla bla bla bla bla bla bla bla bla bla bla bla bla bla bla bla bla bla bla bla bla bla bla bla bla bla bla bla bla bla bla bla bla bla bla bla bla bla.
\\


bla bla bla bla bla bla bla bla bla bla bla bla bla bla bla bla bla bla bla bla bla bla bla bla bla bla bla bla bla bla bla bla bla bla bla bla bla bla bla bla bla bla bla bla bla bla bla bla bla bla bla bla bla bla bla bla bla bla bla bla bla bla bla bla bla bla bla bla bla bla bla bla bla bla bla bla bla bla bla bla bla bla bla bla bla bla bla bla bla bla bla bla.

bla bla bla bla bla bla bla bla bla bla bla bla bla bla bla bla bla bla bla bla bla bla bla bla bla bla bla bla bla bla bla bla bla bla bla bla bla bla bla bla bla bla bla bla bla bla bla bla bla bla bla bla bla bla bla bla bla bla bla bla bla bla bla bla bla bla bla bla bla bla bla bla bla bla bla bla bla bla bla bla bla bla bla bla bla bla bla bla bla bla bla bla.
\\


bla bla bla bla bla bla bla bla bla bla bla bla bla bla bla bla bla bla bla bla bla bla bla bla bla bla bla bla bla bla bla bla bla bla bla bla bla bla bla bla bla bla bla bla bla bla bla bla bla bla bla bla bla bla bla bla bla bla bla bla bla bla bla bla bla bla bla bla bla bla bla bla bla bla bla bla bla bla bla bla bla bla bla bla bla bla bla bla bla bla bla bla.

bla bla bla bla bla bla bla bla bla bla bla bla bla bla bla bla bla bla bla bla bla bla bla bla bla bla bla bla bla bla bla bla bla bla bla bla bla bla bla bla bla bla bla bla bla bla bla bla bla bla bla bla bla bla bla bla bla bla bla bla bla bla bla bla bla bla bla bla bla bla bla bla bla bla bla bla bla bla bla bla bla bla bla bla bla bla bla bla bla bla bla bla.
\\


bla bla bla bla bla bla bla bla bla bla bla bla bla bla bla bla bla bla bla bla bla bla bla bla bla bla bla bla bla bla bla bla bla bla bla bla bla bla bla bla bla bla bla bla bla bla bla bla bla bla bla bla bla bla bla bla bla bla bla bla bla bla bla bla bla bla bla bla bla bla bla bla bla bla bla bla bla bla bla bla bla bla bla bla bla bla bla bla bla bla bla bla.

bla bla bla bla bla bla bla bla bla bla bla bla bla bla bla bla bla bla bla bla bla bla bla bla bla bla bla bla bla bla bla bla bla bla bla bla bla bla bla bla bla bla bla bla bla bla bla bla bla bla bla bla bla bla bla bla bla bla bla bla bla bla bla bla bla bla bla bla bla bla bla bla bla bla bla bla bla bla bla bla bla bla bla bla bla bla bla bla bla bla bla bla.
\\


bla bla bla bla bla bla bla bla bla bla bla bla bla bla bla bla bla bla bla bla bla bla bla bla bla bla bla bla bla bla bla bla bla bla bla bla bla bla bla bla bla bla bla bla bla bla bla bla bla bla bla bla bla bla bla bla bla bla bla bla bla bla bla bla bla bla bla bla bla bla bla bla bla bla bla bla bla bla bla bla bla bla bla bla bla bla bla bla bla bla bla bla.

bla bla bla bla bla bla bla bla bla bla bla bla bla bla bla bla bla bla bla bla bla bla bla bla bla bla bla bla bla bla bla bla bla bla bla bla bla bla bla bla bla bla bla bla bla bla bla bla bla bla bla bla bla bla bla bla bla bla bla bla bla bla bla bla bla bla bla bla bla bla bla bla bla bla bla bla bla bla bla bla bla bla bla bla bla bla bla bla bla bla bla bla.
\\


\textbf{Keywords : } word, word, word, word, word, word, word, word, word, word, word, word, word, word, word, word, word, word, word, word, word, word, ...

%%%%%%%%%% Abstract %%%%%%%%%%
\cleardoublepage

\chapter*{Abstract}

\thispagestyle{empty}



TODO
\\

\textbf{Keywords : } Sopra Group, Microsoft, VB.NET, Silverlight, SQL Server

%%%%%%%%%% Table des figures %%%%%%%%%%
\cleardoublepage
\pagestyle{empty}
\listoffigures
\thispagestyle{empty}

%%%%%%%%%% Table des matières %%%%%%%%%%
\cleardoublepage
\pagestyle{empty}
\tableofcontents
\thispagestyle{empty}

%%%%%%%%%% Corps du rapport %%%%%%%%%%
\cleardoublepage
\pagestyle{corps}
\pagenumbering{arabic} % pagination en chiffres arabes

%%%%%%%%%% Introduction %%%%%%%%%%
\cleardoublepage

\chapter*{Introduction}

\addcontentsline{toc}{chapter}{Introduction}
\markboth{Introduction}{}


%%%%%%%%%%%%%%%%%%%%%%%%%%%%%%%%%%%%%%%%%%%%%%%%%%%%%%%%%%%%%%%%%%%%%%%%%%%
%%%%%%%%%% Context %%%%%%%%%%%%%%%%%%%%%%%%%%%%%%%%%%%%%%%%%%%%%%%%%%%%%%%%
%%%%%%%%%%%%%%%%%%%%%%%%%%%%%%%%%%%%%%%%%%%%%%%%%%%%%%%%%%%%%%%%%%%%%%%%%%%

%%%%%%%%%% Contexte général %%%%%%%%%%%%%%%%%%%%%%%%%%%%%%%%%%%%%%%%%%%%%%%

Dans de nombreux contextes le gain de temps est le maître mot. Le domaine professionnel ne fait pas exception à la règle, imposant une productivité de plus en plus élevée. Ainsi l'utilisation d'outils de plus en plus performants permet de minimiser le temps passé à effectuer une tache.

%%%%%%%%%% Context de l'entreprise %%%%%%%%%%%%%%%%%%%%%%%%%%%%%%%%%%%%%%%%

Les technologies évoluent au fil des années, imposant la mise à jour des anciens outils pour éviter toute faille de sécurité, améliorer l'ergonomie, mais aussi la simplicité d'utilisation.
\\

%%%%%%%%%%%%%%%%%%%%%%%%%%%%%%%%%%%%%%%%%%%%%%%%%%%%%%%%%%%%%%%%%%%%%%%%%%%
%%%%%%%%%% Problème %%%%%%%%%%%%%%%%%%%%%%%%%%%%%%%%%%%%%%%%%%%%%%%%%%%%%%%
%%%%%%%%%%%%%%%%%%%%%%%%%%%%%%%%%%%%%%%%%%%%%%%%%%%%%%%%%%%%%%%%%%%%%%%%%%%

Lorsqu'un logiciel existe depuis plus de dix ans, il est souvent nécessaire de le faire évoluer pour subvenir aux besoins actuels. Mais lorsque celui-ci a été développé par une personne ne possédant pas des compétences en informatique, il est quasiment impossible de le faire évoluer. La meilleure solution est donc de produire une nouvelle solution qui subviendra aux besoins actuels mais permettant l'ajout de nouvelles fonctionnalités.
\\

%%%%%%%%%%%%%%%%%%%%%%%%%%%%%%%%%%%%%%%%%%%%%%%%%%%%%%%%%%%%%%%%%%%%%%%%%%%
%%%%%%%%%% Objectif %%%%%%%%%%%%%%%%%%%%%%%%%%%%%%%%%%%%%%%%%%%%%%%%%%%%%%%
%%%%%%%%%%%%%%%%%%%%%%%%%%%%%%%%%%%%%%%%%%%%%%%%%%%%%%%%%%%%%%%%%%%%%%%%%%%

L'objet de cette étude sera d'étudier les besoins du client, puis d'implémenter la nouvelle solution. Celle-ci devra remplir les mêmes fonctions que la solution actuelle, en permettant l'ajout de nouvelles.
\\

%%%%%%%%%%%%%%%%%%%%%%%%%%%%%%%%%%%%%%%%%%%%%%%%%%%%%%%%%%%%%%%%%%%%%%%%%%%
%%%%%%%%%% Démarche et plan %%%%%%%%%%%%%%%%%%%%%%%%%%%%%%%%%%%%%%%%%%%%%%%
%%%%%%%%%%%%%%%%%%%%%%%%%%%%%%%%%%%%%%%%%%%%%%%%%%%%%%%%%%%%%%%%%%%%%%%%%%%

%%%%%%%%%% Démarche %%%%%%%%%%%%%%%%%%%%%%%%%%%%%%%%%%%%%%%%%%%%%%%%%%%%%%%

La construction d'une première machine virtuelle complète permettra de tester et de comparer les différentes techniques possibles et utilisables dans la solution finale.
Une fois ces techniques définies, elles seront optimisées et intégrées à une seconde machine virtuelle minimale pour obtenir les meilleurs résultats possibles.
Enfin, l'intégration de la solution dans une interface graphique permettra l'utilisation pratique et efficace d'outils, à l'origine non exécutables.


%%%%%%%%%% Plan %%%%%%%%%%%%%%%%%%%%%%%%%%%%%%%%%%%%%%%%%%%%%%%%%%%%%%%%%%%

Tout d'abord, j'analyserai précisément le sujet, en présentation l'entreprise, la virtualisation, ainsi que la solution envisagée.
Dans un deuxième temps je développerai la démarche suivie pour parvenir à la solution finale du projet.
Pour terminer, je présenterai le développement et la solution obtenue.
\newpage
\thispagestyle{empty} % Bug : entête ce la liste des figures...

%%%%%%%%%% Chapitre 1 %%%%%%%%%%
\cleardoublepage

\chapter{Introduction de l'étude}

%%%%%%%%%%%%%%%%%%%%%%%%%%%%%%%%%%%%%%%%%%%%%%%%%%%%%%%%%%%%%%%%%%%%%%%%%%%
%%%%%%%%%%%%%%%%%%%%%%%%%%%%%%%%%%%%%%%%%%%%%%%%%%%%%%%%%%%%%%%%%%%%%%%%%%%
%%%%%%%%%%%%%%%%%%%%%%%%%%%%%%%%%%%%%%%%%%%%%%%%%%%%%%%%%%%%%%%%%%%%%%%%%%%
%%%%%%%%%%%%%%%%%%%%%%%%%%%%%%%%%%%%%%%%%%%%%%%%%%%%%%%%%%%%%%%%%%%%%%%%%%%
%%%%%%%%%%%%%%%%%%%%%%%%%%%%%%%%%%%%%%%%%%%%%%%%%%%%%%%%%%%%%%%%%%%%%%%%%%%

\section{Présentation de l'entreprise}

%%%%%%%%%%%%%%%%%%%%%%%%%%%%%%%%%%%%%%%%%%%%%%%%%%%%%%%%%%%%%%%%%%%%%%%%%%%
%%%%%%%%%%%%%%%%%%%%%%%%%%%%%%%%%%%%%%%%%%%%%%%%%%%%%%%%%%%%%%%%%%%%%%%%%%%
%%%%%%%%%%%%%%%%%%%%%%%%%%%%%%%%%%%%%%%%%%%%%%%%%%%%%%%%%%%%%%%%%%%%%%%%%%%

\subsection{Historique}

\textit{Sopra Group} est une Société de Service en Ingénierie Informatique, couramment appelé SSII.
Créée en 1968 par Pierre PASQUIER et François ODIN, il s'agit d'une des plus anciennes et importantes SSII.
La société s'est d'abord révélée au niveau national grâce à de grands projets dans les domaines bancaires et avec le Ministère de l'Intérieur, avant de s'implanter au niveau européen et mondial.
Avec plus de 14.300 collaborateurs en 2012, Sopra Group réalise un chiffre d'affaires supérieur à 1,2 milliards d'euros, principalement dans le domaine du conseil et des services en France.

%%%%%%%%%%%%%%%%%%%%%%%%%%%%%%%%%%%%%%%%%%%%%%%%%%%%%%%%%%%%%%%%%%%%%%%%%%%
%%%%%%%%%%%%%%%%%%%%%%%%%%%%%%%%%%%%%%%%%%%%%%%%%%%%%%%%%%%%%%%%%%%%%%%%%%%
%%%%%%%%%%%%%%%%%%%%%%%%%%%%%%%%%%%%%%%%%%%%%%%%%%%%%%%%%%%%%%%%%%%%%%%%%%%

\subsection{Agence de Clermont-Ferrand}

Michelin est le plus gros employeur de la région Auvergne, mais fait aussi appel à de nombreuses sociétés extérieures comme les sociétés de services.
Dû au nombre croissant de collaborateurs travaillant chez Michelin, la division Rhône-Alpes a créé l'agence de Clermont-Ferrand pour être au plus proche du client.

L'agence de Clermont-Ferrand travaille aussi sur plusieurs autres projets importants.
TODO: verbe Le pôle identité mène les projets Sides et SITI spécialisés dans la gestion de titres (documents d'identité, passeport, cartes à puce, \ldots), et travaille pour une quinzaine de gouvernements.
Le groupe Limagrain, dont le siège social est situé Chappes, fait actuellement appel aux services de Sopra Group pour la réalisation du projet Boss.

%%%%%%%%%%%%%%%%%%%%%%%%%%%%%%%%%%%%%%%%%%%%%%%%%%%%%%%%%%%%%%%%%%%%%%%%%%%
%%%%%%%%%%%%%%%%%%%%%%%%%%%%%%%%%%%%%%%%%%%%%%%%%%%%%%%%%%%%%%%%%%%%%%%%%%%
%%%%%%%%%%%%%%%%%%%%%%%%%%%%%%%%%%%%%%%%%%%%%%%%%%%%%%%%%%%%%%%%%%%%%%%%%%%
%%%%%%%%%%%%%%%%%%%%%%%%%%%%%%%%%%%%%%%%%%%%%%%%%%%%%%%%%%%%%%%%%%%%%%%%%%%
%%%%%%%%%%%%%%%%%%%%%%%%%%%%%%%%%%%%%%%%%%%%%%%%%%%%%%%%%%%%%%%%%%%%%%%%%%%

\section{Étude du problème}

%%%%%%%%%%%%%%%%%%%%%%%%%%%%%%%%%%%%%%%%%%%%%%%%%%%%%%%%%%%%%%%%%%%%%%%%%%%
%%%%%%%%%%%%%%%%%%%%%%%%%%%%%%%%%%%%%%%%%%%%%%%%%%%%%%%%%%%%%%%%%%%%%%%%%%%
%%%%%%%%%%%%%%%%%%%%%%%%%%%%%%%%%%%%%%%%%%%%%%%%%%%%%%%%%%%%%%%%%%%%%%%%%%%

\subsection{Besoin de l'utilisateur}

Durant l'été, saison des récoltes de céréales, Limagrain emploie de nombreux travailleurs.
Il s'agit principalement de contrats saisonniers ou à durée déterminée.
\\

Le service de recrutement du groupe Limagrain a besoin d'enregistrer les informations de chacun de ses employés pour ne pas avoir à les ressaisir lors des prochains contrats.
En effet, le service emploie de préférence d'anciens candidats.
De plus, les contrats sont parfois renouvelés plusieurs fois durant la saison ou chaque année.

L'Urssaf, union de recouvrement des cotisations de sécurité sociale et d'allocations familiales, impose la saisie d'un document : la DAPE, déclaration préalable à l'embauche.
Pour chaque recrutement, il est donc nécessaire de remplir ce document avec les informations du candidat ainsi que du contrat.
Le client désire donc pouvoir générer automatiquement ces documents, lui évitant ainsi de les saisir manuellement un par un et par contrat.

Pour faciliter le recrutement d'anciens candidats, ou la recherche de candidats disponibles, le client désire avoir différentes fonctionnalités de recherche.
Il peut s'agir, par exemple, de la recherche de candidats disponibles à certaines périodes, ou les candidats ayant un contrat en cours et qui n'ont pas fourni toutes les pièces justificatives.

Enfin pour pouvoir augmenter leur productivité, il sera nécessaire de produire une application ergonomique et facile d'utilisation permettant la saisie de nombreuses données le plus rapidement possible.
Limagrain reçoit en effet plus de 1.000 candidatures en moins d'un mois dont une grande partie effectue des contrats.

%%%%%%%%%%%%%%%%%%%%%%%%%%%%%%%%%%%%%%%%%%%%%%%%%%%%%%%%%%%%%%%%%%%%%%%%%%%
%%%%%%%%%%%%%%%%%%%%%%%%%%%%%%%%%%%%%%%%%%%%%%%%%%%%%%%%%%%%%%%%%%%%%%%%%%%
%%%%%%%%%%%%%%%%%%%%%%%%%%%%%%%%%%%%%%%%%%%%%%%%%%%%%%%%%%%%%%%%%%%%%%%%%%%

\subsection{Solution actuelle}

Le service de recrutement utilisait une petite application Access développée à partir d'un document Excel, de Microsoft Office.
Ce document comporte une base de données ainsi qu'une interface de saisie.
\\

Cette solution comporte de nombreux inconvénients.

Tout d'abord, la saisie des données est fastidieuse car l'ergonomie est très sommaire : les champs sont disposés de manière désordonnée sur la page.
De plus, pour la saisie d'un établissement par exemple, il était nécessaire de saisir son code, sans pouvoir voir directement son nom ou toute autre information.

Les fonctionnalités sont limitées, et se limitent principalement à la saisie.
Les attentes de l'utilisateur sont donc très loin de ce que propose l'outil actuel.

Enfin, il est impossible de travailler en simultané avec cet outil car les sauvegardes multiples ne sont pas prises en compte.

%%%%%%%%%%%%%%%%%%%%%%%%%%%%%%%%%%%%%%%%%%%%%%%%%%%%%%%%%%%%%%%%%%%%%%%%%%%
%%%%%%%%%%%%%%%%%%%%%%%%%%%%%%%%%%%%%%%%%%%%%%%%%%%%%%%%%%%%%%%%%%%%%%%%%%%
%%%%%%%%%%%%%%%%%%%%%%%%%%%%%%%%%%%%%%%%%%%%%%%%%%%%%%%%%%%%%%%%%%%%%%%%%%%

\subsection{Solution envisagée}

En étudiant la solution actuelle, les chefs de projet et architectes ont décidé de développer une nouvelle solution, en raison de l'impossibilité de maintenir et faire évoluer l'ancienne.
Cette nouvelle solution devra répondre aux besoins de l'utilisateur, en corrigeant les inconvénients de l'ancienne ainsi qu'en apportant de nouvelles fonctionnalités.
\\

Il est tout d'abord nécessaire de changer de base de données pour se tourner vers un système plus performant et centralisé.
Un nouveau schéma de données adapté aux nouveaux besoins permettra de mieux structurer ces données.

Pour améliorer l'ergonomie de l'application, la nouvelle solution sera un \textit{client léger}.
Dans ce projet nous ne rencontrerons pas les inconvénients qu'offrent cette solution, comme les performances ou les problèmes qui affecteront tous les utilisateurs.
Par contre, un simple site web offre les avantages d'un déploiement unique, d'une maintenance simple, car cela ne se fait que sur le serveur et non sur chacun des ordinateurs des utilisateurs.

%%%%%%%%%% Chapitre 2 %%%%%%%%%%
\cleardoublepage

\chapter{Sides}

%%%%%%%%%%%%%%%%%%%%%%%%%%%%%%%%%%%%%%%%%%%%%%%%%%%%%%%%%%%%%%%%%%%%%%%%%%%
%%%%%%%%%%%%%%%%%%%%%%%%%%%%%%%%%%%%%%%%%%%%%%%%%%%%%%%%%%%%%%%%%%%%%%%%%%%
%%%%%%%%%%%%%%%%%%%%%%%%%%%%%%%%%%%%%%%%%%%%%%%%%%%%%%%%%%%%%%%%%%%%%%%%%%%
%%%%%%%%%%%%%%%%%%%%%%%%%%%%%%%%%%%%%%%%%%%%%%%%%%%%%%%%%%%%%%%%%%%%%%%%%%%
%%%%%%%%%%%%%%%%%%%%%%%%%%%%%%%%%%%%%%%%%%%%%%%%%%%%%%%%%%%%%%%%%%%%%%%%%%%

\section{Présentation}

\textit{Sides} est un projet de gestion de titres développé par Sopra Group, plus particulièrement par l'agence de Clermont Ferrand.
Il permet de produire des passeports ou des cartes d'identités pour une dizaine de différents gouvernement comme les Philippines, la Belgique, Monaco, \ldots
\\

Pour chacun des pays et titres, une application a été développée.
Plutôt que de redévelopper la même application, un nouveau projet a vu le jour : \textit{SITI}.
Cette solution pourra ensuite être vendue à divers pays sans avoir à la ré-implémenter à nouveau.

Actuellement l'équipe Sides n'effectue que de la tiers maintenance applicative (TMA) sur les différentes applications existantes.
L'objectif est de corriger les bugs éventuels ou d'ajouter de nouvelles fonctionnalités.

%%%%%%%%%%%%%%%%%%%%%%%%%%%%%%%%%%%%%%%%%%%%%%%%%%%%%%%%%%%%%%%%%%%%%%%%%%%
%%%%%%%%%%%%%%%%%%%%%%%%%%%%%%%%%%%%%%%%%%%%%%%%%%%%%%%%%%%%%%%%%%%%%%%%%%%
%%%%%%%%%%%%%%%%%%%%%%%%%%%%%%%%%%%%%%%%%%%%%%%%%%%%%%%%%%%%%%%%%%%%%%%%%%%
%%%%%%%%%%%%%%%%%%%%%%%%%%%%%%%%%%%%%%%%%%%%%%%%%%%%%%%%%%%%%%%%%%%%%%%%%%%
%%%%%%%%%%%%%%%%%%%%%%%%%%%%%%%%%%%%%%%%%%%%%%%%%%%%%%%%%%%%%%%%%%%%%%%%%%%

\section{Chargement d'empreintes}

%%%%%%%%%%%%%%%%%%%%%%%%%%%%%%%%%%%%%%%%%%%%%%%%%%%%%%%%%%%%%%%%%%%%%%%%%%%
%%%%%%%%%%%%%%%%%%%%%%%%%%%%%%%%%%%%%%%%%%%%%%%%%%%%%%%%%%%%%%%%%%%%%%%%%%%
%%%%%%%%%%%%%%%%%%%%%%%%%%%%%%%%%%%%%%%%%%%%%%%%%%%%%%%%%%%%%%%%%%%%%%%%%%%

\subsection{Contexte}

L'application Sides à destination des Philippines permet la gestion de passeports.
Ces passeports nécessitent la lecture des empreintes du porteur pour être valides.
Lorsque les porteurs sont des personnes diplomatiques ou haut placées, celles-ci désireraient ne pas perdre de temps lors de création ou renouvellement de titres.

Pour satisfaire ces exigences, une personne du gouvernement munie d'un ordinateur portable, de l'application Sides et du périphérique va effectuer la saisie des empreintes sur le lieu désiré : ville plus proche, lieu de travail, \ldots

%%%%%%%%%%%%%%%%%%%%%%%%%%%%%%%%%%%%%%%%%%%%%%%%%%%%%%%%%%%%%%%%%%%%%%%%%%%
%%%%%%%%%%%%%%%%%%%%%%%%%%%%%%%%%%%%%%%%%%%%%%%%%%%%%%%%%%%%%%%%%%%%%%%%%%%
%%%%%%%%%%%%%%%%%%%%%%%%%%%%%%%%%%%%%%%%%%%%%%%%%%%%%%%%%%%%%%%%%%%%%%%%%%%

\subsection{Contrôle de la date}

La première version de cette fonctionnalité permettait l'importation des empreintes les plus récentes présentes dans la base de données pour créer un nouveau passeport.
Les empreintes peuvent venir d'une ancienne demande ou avoir été saisies pour une personne diplomatique comme expliqué précédemment.

Pour sécuriser l'application et le chargement d'anciennes empreintes, l'évolution de la fonctionnalité a pour objectif de contrôler la date d'importation des empreintes dans la base de données.
\\

Pour réaliser ce contrôle, j'ai dû étudier le fonctionnement de l'application pour comprendre le processus de chargement des empreintes dans une nouvelle demande de passeport.

%%%%%%%%%%%%%%%%%%%%%%%%%%%%%%%%%%%%%%%%%%%%%%%%%%%%%%%%%%%%%%%%%%%%%%%%%%%
%%%%%%%%%%%%%%%%%%%%%%%%%%%%%%%%%%%%%%%%%%%%%%%%%%%%%%%%%%%%%%%%%%%%%%%%%%%
%%%%%%%%%%%%%%%%%%%%%%%%%%%%%%%%%%%%%%%%%%%%%%%%%%%%%%%%%%%%%%%%%%%%%%%%%%%
%%%%%%%%%%%%%%%%%%%%%%%%%%%%%%%%%%%%%%%%%%%%%%%%%%%%%%%%%%%%%%%%%%%%%%%%%%%
%%%%%%%%%%%%%%%%%%%%%%%%%%%%%%%%%%%%%%%%%%%%%%%%%%%%%%%%%%%%%%%%%%%%%%%%%%%

\section{Les certificats}

%%%%%%%%%%%%%%%%%%%%%%%%%%%%%%%%%%%%%%%%%%%%%%%%%%%%%%%%%%%%%%%%%%%%%%%%%%%
%%%%%%%%%%%%%%%%%%%%%%%%%%%%%%%%%%%%%%%%%%%%%%%%%%%%%%%%%%%%%%%%%%%%%%%%%%%
%%%%%%%%%%%%%%%%%%%%%%%%%%%%%%%%%%%%%%%%%%%%%%%%%%%%%%%%%%%%%%%%%%%%%%%%%%%

\subsection{Contexte}

La gestion des identités de personnes, au niveau national, est un domaine très sensible qui requière une sécurité élevée des systèmes informatiques.
La solution permettant de sécuriser les données consiste à chiffrer ou signer les données échangées, garantissant ainsi que les données ne pourront être découverte ou qu'elles n'ont pas été modifiée ou envoyées par une personne tiers. Ce chiffrement s'effectue grâce à des certificats.

%%%%%%%%%%%%%%%%%%%%%%%%%%%%%%%%%%%%%%%%%%%%%%%%%%%%%%%%%%%%%%%%%%%%%%%%%%%
%%%%%%%%%%%%%%%%%%%%%%%%%%%%%%%%%%%%%%%%%%%%%%%%%%%%%%%%%%%%%%%%%%%%%%%%%%%
%%%%%%%%%%%%%%%%%%%%%%%%%%%%%%%%%%%%%%%%%%%%%%%%%%%%%%%%%%%%%%%%%%%%%%%%%%%

\subsubsection{Le chiffrement}

Pour assurer que personne ne puisse découvrir le message échangé entre deux personnes il est nécessaire de les chiffrer.

Cette méthode utilise deux entités : un couple de clés composé d'une clé publique ($K$) et d'une clé privée ($K'$), et un algorithme mathématique composé d'une fonction ($F$) et de sa réciproque ($F'$).
Le chiffrement consiste à utiliser une clé dans l'algorithme, et le déchiffrement à utiliser la fonction inverse avec l'autre clé pour retrouver les données d'origine, comme le résume la fonction suivant :
\[
F_K[\ F_{K'}(M)\ ] = M = F'_{K'}[\ F_K(M)\ ]
\]
\\

Le fonctionnement de l'échange d'un message chiffré est le suivant, schématisé par la figure \ref{chiffrement} :
\begin{enumerate}
	\item Bob possède une paire de clé publique/privée, et Alice désire envoyer un message à Bob ;
	\item Bob envoie sa clé publique à Alice ;
	\item Alice chiffre son message grâce à cette clé publique ;
	\item Alice envoie ensuite son message chiffré à Bob ;
	\item Bob déchiffre le message chiffré grâce à sa clé privée. Personne d'autre ne peut le faire car il est le seul à posséder la clé privée.
\end{enumerate}
\begin{figure}[!h]
	\center
	\includegraphics[width=0.7\textwidth]{img/chiffrement.png}
	\caption{Échange d'une message chiffré}
	\label{chiffrement}
\end{figure}
~~\\

La sécurité est basée sur deux principes : le chiffrement et la confidentialité.

Si l'algorithme de chiffrement est trop faible alors il sera possible de déchiffrer les données par calcul mathématiques.
De plus, si la clé de chiffrement est trop petite alors la méthode "brute force", consistant à essayer toutes les combinaisons possibles, permettra aussi le déchiffrement

Si la clé privée du certificat est découverte alors les données seront instantanément déchiffrées, ainsi que les futurs messages.
Il est donc nécessaire de renouveler les clés régulièrement.

%%%%%%%%%%%%%%%%%%%%%%%%%%%%%%%%%%%%%%%%%%%%%%%%%%%%%%%%%%%%%%%%%%%%%%%%%%%

\subsubsection{La signature}
\label{La signature}

La signature numérique permet de vérifier l'authenticité et la non-modification d'un message.

Cette méthode utilise une fonction hachage qui a pour objectif de créer une empreinte à taille fixe.
Les algorithmes de hachage les plus utilisés sont MP5 et SHA.
\\

Le fonctionnement est le suivant, schématisé par la figure \ref{signature} :
\begin{enumerate}
	\item Bob hache le message, puis chiffre ce hash avec sa clé privée, ce qui produit la signature du message ;
	\item Bob envoie à Alice le message, la clé publique et la signature ;
	\item Alice hache le message, et déchiffre la signature du message grâce à la clé publique. Ensuite elle compare les deux valeurs, qui seront égales si le message n'a pas été modifié et a été chiffré avec la clé privée associée à la clé publique.
\end{enumerate}
\begin{figure}[!h]
	\center
	\includegraphics[width=0.7\textwidth]{img/signature.png}
	\caption{Échange d'un message signé}
	\label{signature}
\end{figure}

%%%%%%%%%%%%%%%%%%%%%%%%%%%%%%%%%%%%%%%%%%%%%%%%%%%%%%%%%%%%%%%%%%%%%%%%%%%

\subsubsection{Le certificat}

Un certificat est composé de :
\begin{itemize}
	\item une clé publique, et éventuellement la clé privée associée ;
	\item des informations sur le propriétaire ;
	\item une signature.
\end{itemize}

Les certificats sont signés par une autorité de certification, aussi appelé CA pour Certificat Autority, qui permet de garantir l'authenticité des certificats.
Seule l'autorité de certification racine est auto-signée.

%%%%%%%%%%%%%%%%%%%%%%%%%%%%%%%%%%%%%%%%%%%%%%%%%%%%%%%%%%%%%%%%%%%%%%%%%%%

\subsubsection{La PKI}

Une \textit{infrastructure à clés publiques}, aussi appelé par son terme anglais \textit{public key infrastructure}, permet la gestion de grandes quantités de certificats ou clés publiques.
Elle fournie des services de création, de publication et de révocation de certificats.
Cette infrastructure est principalement dans les systèmes informatiques utilisant de nombreux certificats.

%%%%%%%%%%%%%%%%%%%%%%%%%%%%%%%%%%%%%%%%%%%%%%%%%%%%%%%%%%%%%%%%%%%%%%%%%%%
%%%%%%%%%%%%%%%%%%%%%%%%%%%%%%%%%%%%%%%%%%%%%%%%%%%%%%%%%%%%%%%%%%%%%%%%%%%
%%%%%%%%%%%%%%%%%%%%%%%%%%%%%%%%%%%%%%%%%%%%%%%%%%%%%%%%%%%%%%%%%%%%%%%%%%%

\subsection{EJBCA}

%%%%%%%%%%%%%%%%%%%%%%%%%%%%%%%%%%%%%%%%%%%%%%%%%%%%%%%%%%%%%%%%%%%%%%%%%%%

\subsubsection{Utilisation dans Sides}

En production, le projet Sides utilise de nombreux certificats pour sécuriser les données.
Pour cela une PKI est utilisée permettant la gestion des nombreux certificats.

Lorsque l'on souhaite tester l'application avant la mise en production, il est nécessaire de pouvoir produire des certificats.
La mise en place d'une PKI peut s'avérer fastidieuse à intégrer et paramétrer pour fournir les mêmes services que la PKI de production.

Pour cela une PKI a été déployée sur les serveurs de tests, fournissant ainsi les fonctionnalités minimales utiles, notamment la production de certificats.
La PKI open-source EJBCA est utilisée par l'équipe Sides et le client pour gérer des certificats de test.

%%%%%%%%%%%%%%%%%%%%%%%%%%%%%%%%%%%%%%%%%%%%%%%%%%%%%%%%%%%%%%%%%%%%%%%%%%%

\subsubsection{Documentation}

Mon second objectif dans l'équipe Sides a été de rédiger une documentation de production de certificats en utilisant EJBCA.
Ces certificats doivent respecter les spécifications d'ICAO\footnote{ICAO - Site web : \url{http://www.icao.int}}, International Civil Aviation Organization, qui définissent les procédures et conseils destinés aux états et fournisseurs de solutions de PKI.

La documentation sera utilisée par le client pour produire ses certificats de tests.
Elle devait donc être la plus précise possible, détaillant toutes les étapes de la création, en spécifiant chacun des paramètres et options.
\\

Un profil définie les paramètres que les certificats, qui l'implémentent, devront définir.
J'ai tout d'abord créé deux profils qui seront utilisés pour le CSCA et le DS.

Le \textit{CSCA}, pour Country Signing Certificate Autority, est le certificat de plus haut niveau, utilisé par le le pays.
Il s'agit plus du plus critique car de lui dépend tous les autres certificats utilisés.
Dans certains pays, celui-ci est stocké dans un ordinateur portable situé dans un coffre fort, qui est ouvert en cas de besoin notamment lors du renouvellement des certificats.

Le \textit{DS}, pour Document Signer, est utilisé pour signer les documents sécurisés.
Il est situé au second niveau dans la hiérarchie des certificats, et est signé par le certificat racine CSCA.

Des certificats, composés d'une clé publique et une clé privée, peuvent ensuite être créés.
Ils peuvent être téléchargés sous le format PKCS\#12 qui est un fichier à l'extension ".p12".
L'utilisateur peut ensuite l'utiliser ou l'installer sur son environnement.

La figure \ref{hierarchie_CSCA_DS} représente la hiérarchie des différents certificats utilisés.
\begin{figure}[!h]
	\center
	\includegraphics[width=0.3\textwidth]{img/hierarchie_CSCA_DS.png}
	\caption{Hiérarchie des certificats}
	\label{hierarchie_CSCA_DS}
\end{figure}

%%%%%%%%%%%%%%%%%%%%%%%%%%%%%%%%%%%%%%%%%%%%%%%%%%%%%%%%%%%%%%%%%%%%%%%%%%%
%%%%%%%%%%%%%%%%%%%%%%%%%%%%%%%%%%%%%%%%%%%%%%%%%%%%%%%%%%%%%%%%%%%%%%%%%%%
%%%%%%%%%%%%%%%%%%%%%%%%%%%%%%%%%%%%%%%%%%%%%%%%%%%%%%%%%%%%%%%%%%%%%%%%%%%

\subsection{Le SOD}

Le \textit{SOD}, Document Security Object, est la signature d'un document.
Cette signature est générée en utilisant la clé privée d'un certificat de type DS, présenté dans la section précédente.

Dans ce cadre j'ai travaillé sur le SOD des cartes à puce, plus particulièrement sur la création de la signature.
Ces cartes à puces contiennent :
\begin{enumerate}
	\item des données, disposées dans seize groupes (DG, pour Data Group), contenant les informations comme l'identité et la photo d'une personne ;
	\item la signature des données appelée SOD, créée à partir des différents DG et d'un certificat.
\end{enumerate}

Le SOD permet de vérifier que les données de la carte à puce n'ont pas été modifiées ou altérées.
Le processus de vérification est similaire à la vérification de la signature d'un message présenté dans la section \ref{La signature} :
\begin{itemize}
	\item On déchiffre le SOD avec la clé publique du DS ;
	\item On hache les données contenues dans les DGs ;
	\item Si les deux valeurs sont égales alors les données sont correctes.
\end{itemize}

%%%%%%%%%%%%%%%%%%%%%%%%%%%%%%%%%%%%%%%%%%%%%%%%%%%%%%%%%%%%%%%%%%%%%%%%%%%

\subsubsection{Documentation de fonctionnement}
\label{Documentation de fonctionnement}

Dans ce contexte, j'ai rédigé une documentation permettant d'expliquer comment créer la signature à partie des différents groupes de données (DG).
La création utilise une application web qui prend en paramètre les valeurs hexadécimales des groupes de données.
La signature est ensuite retournée sous la forme d'un fichier binaire.
Le figure \ref{processus_SOD} schématise le processus de création de la signature SOD.
\begin{figure}[!h]
	\center
	\includegraphics[width=0.2\textwidth]{img/processus_SOD.png}
	\caption{Processus de création du SOD}
	\label{processus_SOD}
\end{figure}

Une page HTML, dont une capture d'écran est présente sur la figure \ref{saisie_DGs}, permet de faciliter la saisie manuelle des données et l'appel au service en proposant respectivement des zones de saisie et un bouton de soumission.
La fenêtre de téléchargement s'affiche ensuite pour proposer d'ouvrir ou de sauvegarder le fichier constituant la signature.
\begin{figure}[!h]
	\center
	\includegraphics[width=0.2\textwidth]{img/saisie_DGs.png}
	\caption{Page HTML de saisie des groupes de données}
	\label{saisie_DGs}
\end{figure}

Il est aussi possible d'appeler le service de manière automatique depuis un programme.
Pour expliquer cela j'ai développé un exemple de script Python permettant de générer une signature.
\begin{lstlisting}[language = sh]
TODO
\end{lstlisting}

%%%%%%%%%%%%%%%%%%%%%%%%%%%%%%%%%%%%%%%%%%%%%%%%%%%%%%%%%%%%%%%%%%%%%%%%%%%

\subsubsection{Application de test}

L'application web de création de signature est développée en JEE (Java Enterprise Edition).
Une \textit{servlet} est une classe java appelée lors d'une requête HTTP et permet de générer dynamiquement des données (généralement des pages HTML).
Le projet web fournie plusieurs servlets permettant d'utiliser plusieurs méthodes de signature, comme la connexion à un HSM (Hardware Security Module) ou l'utilisation de certificat PKCS\#12 décrit précédemment.

Dans l'objectif de fournir au client une application permettant de tester la création de signature SOD à partir d'un certificat PKCS\#12, j'ai modifié le projet web initial pour ne garder que la fonctionnalité correspondante.
L'épuration du code source et la correction de certains bugs ont permis de produire une solution fonctionnelle qui fourni les services de signature de données décrites dans la partie \ref{Documentation de fonctionnement}.

%%%%%%%%%% Chapitre 3 %%%%%%%%%%
\cleardoublepage

\chapter{Résultats - Discussions}



\section{Interface graphique utilisateur}

J'ai été libre de programmer la première version de l'interface graphique, comportant les fonctionnalités demandées par le client. Ce dernier a participé à sa réalisation en effectuant diverses critiques et réagencements des éléments afin d'optimiser l'ergonomie de l'application.

L'application est composée d'une barre latéral sur la partie gauche, proposant un menu. Un menu correspond à un module de l'application, et chaque sous-menu correspond à une fonctionnalité du module. Les écrans peuvent peuvent être de plusieurs type :
\begin{enumerate}
	\item Un écran de recherche, avec 
\end{enumerate}

La figure \ref{TODO} est une capture d'écran de l'application.


\section{Base de données}

Le modèle d

TODO: dépendances manquantes (spécifiées à la dernière minute), importante incomplètes



~~\\-------------

tables de paramétrage

%%%%%%%%%% Conclusion %%%%%%%%%%
\cleardoublepage

\chapter*{Conclusion}

\addcontentsline{toc}{chapter}{Conclusion}
\markboth{Conclusion}{}



% Rapppel du sujet
Ce projet avait pour objectif le développement d'une nouvelle application de recrutement à destination du service recrutement du groupe Limagrain.
L'application devait reprendre les fonctionnalités de l'ancien outil, en y ajoutant de nouvelles fonctionnalités pour satisfaire aux besoins de l'utilisateur.

% Etude terminée ?
La première version de l'application est actuellement terminée et en production chez le client.
Les fonctionnalités prévues dans le premier lot ont été implémentées.
Un second lot est prévu pour le mois de septembre, ajoutant de nouvelles fonctionnalités et appliquant les modifications éventuelles.

% Améliorations possibles
Certaines améliorations sont possibles, notamment au niveau ergonomie, mais cela nécessiterait l'intervention d'une personne spécialisée.
\\

% Difficultés rencontrées
Les principaux problèmes rencontrés concernaient la découverte des différentes technologies utilisées, notamment le socle utilisé sur lequel se base les différents projets.

% Ce que ça m'a apporté
Ce projet m'a permis d'améliorer ma communication et la compréhension du besoin, du fait de la proximité directe avec le client.
De plus j'ai pu découvrir de nouvelles technologies approfondissant ainsi mes connaissances.
J'ai pu ainsi découvrir le métier d'ingénieur avec ses différents aspects fonctionnels et techniques.

\clearpage
\pagestyle{empty}
\cleardoublepage

%%%%%%%%%% Bibliographie %%%%%%%%%%
%\cleardoublepage

\chapter*{Webographie}

\thispagestyle{empty}



TODO

\end{document}
