\cleardoublepage

\chapter*{Introduction}

\addcontentsline{toc}{chapter}{Introduction}
\markboth{Introduction}{}


%%%%%%%%%%%%%%%%%%%%%%%%%%%%%%%%%%%%%%%%%%%%%%%%%%%%%%%%%%%%%%%%%%%%%%%%%%%
%%%%%%%%%% Context %%%%%%%%%%%%%%%%%%%%%%%%%%%%%%%%%%%%%%%%%%%%%%%%%%%%%%%%
%%%%%%%%%%%%%%%%%%%%%%%%%%%%%%%%%%%%%%%%%%%%%%%%%%%%%%%%%%%%%%%%%%%%%%%%%%%

%%%%%%%%%% Contexte général %%%%%%%%%%%%%%%%%%%%%%%%%%%%%%%%%%%%%%%%%%%%%%%

Dans de nombreux contextes le gain de temps est le maître mot. Le domaine professionnel ne fait pas exception à la règle, imposant une productivité de plus en plus élevée. Ainsi l'utilisation d'outils de plus en plus performants permet de minimiser le temps passé à effectuer une tache.

%%%%%%%%%% Context de l'entreprise %%%%%%%%%%%%%%%%%%%%%%%%%%%%%%%%%%%%%%%%

Les technologies évoluent au fil des années, imposant la mise à jour des anciens outils pour éviter toute faille de sécurité, améliorer l'ergonomie, mais aussi la simplicité d'utilisation.
\\

%%%%%%%%%%%%%%%%%%%%%%%%%%%%%%%%%%%%%%%%%%%%%%%%%%%%%%%%%%%%%%%%%%%%%%%%%%%
%%%%%%%%%% Problème %%%%%%%%%%%%%%%%%%%%%%%%%%%%%%%%%%%%%%%%%%%%%%%%%%%%%%%
%%%%%%%%%%%%%%%%%%%%%%%%%%%%%%%%%%%%%%%%%%%%%%%%%%%%%%%%%%%%%%%%%%%%%%%%%%%

Lorsqu'un logiciel existe depuis plus de dix ans, il est souvent nécessaire de le faire évoluer pour subvenir aux besoins actuels. Mais lorsque celui-ci a été développé par une personne ne possédant pas des compétences en informatique, il est quasiment impossible de le faire évoluer. La meilleure solution est donc de produire une nouvelle solution qui subviendra aux besoins actuels mais permettant l'ajout de nouvelles fonctionnalités.
\\

%%%%%%%%%%%%%%%%%%%%%%%%%%%%%%%%%%%%%%%%%%%%%%%%%%%%%%%%%%%%%%%%%%%%%%%%%%%
%%%%%%%%%% Objectif %%%%%%%%%%%%%%%%%%%%%%%%%%%%%%%%%%%%%%%%%%%%%%%%%%%%%%%
%%%%%%%%%%%%%%%%%%%%%%%%%%%%%%%%%%%%%%%%%%%%%%%%%%%%%%%%%%%%%%%%%%%%%%%%%%%

L'objet de cette étude sera d'étudier les besoins du client, puis d'implémenter la nouvelle solution. Celle-ci devra remplir les mêmes fonctions que la solution actuelle, en permettant l'ajout de nouvelles.
\\

%%%%%%%%%%%%%%%%%%%%%%%%%%%%%%%%%%%%%%%%%%%%%%%%%%%%%%%%%%%%%%%%%%%%%%%%%%%
%%%%%%%%%% Démarche et plan %%%%%%%%%%%%%%%%%%%%%%%%%%%%%%%%%%%%%%%%%%%%%%%
%%%%%%%%%%%%%%%%%%%%%%%%%%%%%%%%%%%%%%%%%%%%%%%%%%%%%%%%%%%%%%%%%%%%%%%%%%%

%%%%%%%%%% Démarche %%%%%%%%%%%%%%%%%%%%%%%%%%%%%%%%%%%%%%%%%%%%%%%%%%%%%%%

La construction d'une première machine virtuelle complète permettra de tester et de comparer les différentes techniques possibles et utilisables dans la solution finale.
Une fois ces techniques définies, elles seront optimisées et intégrées à une seconde machine virtuelle minimale pour obtenir les meilleurs résultats possibles.
Enfin, l'intégration de la solution dans une interface graphique permettra l'utilisation pratique et efficace d'outils, à l'origine non exécutables.


%%%%%%%%%% Plan %%%%%%%%%%%%%%%%%%%%%%%%%%%%%%%%%%%%%%%%%%%%%%%%%%%%%%%%%%%

Tout d'abord, j'analyserai précisément le sujet, en présentation l'entreprise, la virtualisation, ainsi que la solution envisagée.
Dans un deuxième temps je développerai la démarche suivie pour parvenir à la solution finale du projet.
Pour terminer, je présenterai le développement et la solution obtenue.