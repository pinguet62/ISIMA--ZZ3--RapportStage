\cleardoublepage

\chapter*{Introduction}

\addcontentsline{toc}{chapter}{Introduction}
\markboth{Introduction}{}


%%%%%%%%%%%%%%%%%%%%%%%%%%%%%%%%%%%%%%%%%%%%%%%%%%%%%%%%%%%%%%%%%%%%%%%%%%%
%%%%%%%%%% Context %%%%%%%%%%%%%%%%%%%%%%%%%%%%%%%%%%%%%%%%%%%%%%%%%%%%%%%%
%%%%%%%%%%%%%%%%%%%%%%%%%%%%%%%%%%%%%%%%%%%%%%%%%%%%%%%%%%%%%%%%%%%%%%%%%%%

%%%%%%%%%% Contexte général %%%%%%%%%%%%%%%%%%%%%%%%%%%%%%%%%%%%%%%%%%%%%%%

Dans de nombreux contextes le gain de temps est le maître mot. Le domaine professionnel ne fait pas exception à la règle, imposant une productivité de plus en plus élevée. Ainsi l'utilisation d'outils de plus en plus performants permet de minimiser le temps passé à effectuer une tache.

%%%%%%%%%% Context de l'entreprise %%%%%%%%%%%%%%%%%%%%%%%%%%%%%%%%%%%%%%%%

Limagrain utilise un outil un outil de recrutement, développé il y a plus de dix ans par un membre de service ne possédant de compétence en développement. Dû au nombre de recrutements de plus en plus élevé, et  l'augmentation de la charge de travail, Limagrain désire faire évoluer son outil pour faciliter le processus actuel.
\\

%%%%%%%%%%%%%%%%%%%%%%%%%%%%%%%%%%%%%%%%%%%%%%%%%%%%%%%%%%%%%%%%%%%%%%%%%%%
%%%%%%%%%% Problème %%%%%%%%%%%%%%%%%%%%%%%%%%%%%%%%%%%%%%%%%%%%%%%%%%%%%%%
%%%%%%%%%%%%%%%%%%%%%%%%%%%%%%%%%%%%%%%%%%%%%%%%%%%%%%%%%%%%%%%%%%%%%%%%%%%

Le passage des années, ainsi que l'évolution des besoins, impose une mise à niveau permanente des outils informatiques. Mais lorsque celui-ci utilise des technologies trop anciennes, et ne possède pas une architecture évolutive, il est rarement possible d'intervenir. La meilleure solution est donc une ré-implémentation complète de l'outil, amenant alors à une solution structurée qui pourra évoluer facilement aux cours des années.
\\

%%%%%%%%%%%%%%%%%%%%%%%%%%%%%%%%%%%%%%%%%%%%%%%%%%%%%%%%%%%%%%%%%%%%%%%%%%%
%%%%%%%%%% Objectif %%%%%%%%%%%%%%%%%%%%%%%%%%%%%%%%%%%%%%%%%%%%%%%%%%%%%%%
%%%%%%%%%%%%%%%%%%%%%%%%%%%%%%%%%%%%%%%%%%%%%%%%%%%%%%%%%%%%%%%%%%%%%%%%%%%

L'objet de cette étude sera d'étudier les besoins du client, puis d'implémenter la nouvelle solution. Celle-ci devra remplir les mêmes fonctions que la solution actuelle, en permettant l'ajout de nouvelles.
\\

%%%%%%%%%%%%%%%%%%%%%%%%%%%%%%%%%%%%%%%%%%%%%%%%%%%%%%%%%%%%%%%%%%%%%%%%%%%
%%%%%%%%%% Démarche et plan %%%%%%%%%%%%%%%%%%%%%%%%%%%%%%%%%%%%%%%%%%%%%%%
%%%%%%%%%%%%%%%%%%%%%%%%%%%%%%%%%%%%%%%%%%%%%%%%%%%%%%%%%%%%%%%%%%%%%%%%%%%

%%%%%%%%%% Démarche %%%%%%%%%%%%%%%%%%%%%%%%%%%%%%%%%%%%%%%%%%%%%%%%%%%%%%%

L'étude de la solution actuelle et l'étude des besoins de l'utilisateur permettra de modéliser la nouvelle solution. L'implémentation d'une première version de l'application permettra de soulever les problèmes existants et d'émettre les points d'amélioration. Enfin la mise en production et l'importation des anciennes données a permis au client d'utiliser l'application.

%%%%%%%%%% Plan %%%%%%%%%%%%%%%%%%%%%%%%%%%%%%%%%%%%%%%%%%%%%%%%%%%%%%%%%%%

Tout d'abord, j'analyserai précisément le sujet, en présentation l'entreprise, ainsi que la solution actuelle et celle envisagée. Dans un deuxième temps, je présenterai l'environnement de travail et la structure du projet. Pour terminer, je présenterai la solution obtenue et les différents points d'amélioration possibles.