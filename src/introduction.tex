\cleardoublepage

\chapter*{Introduction}

\addcontentsline{toc}{chapter}{Introduction}
\markboth{Introduction}{}


%%%%%%%%%%%%%%%%%%%%%%%%%%%%%%%%%%%%%%%%%%%%%%%%%%%%%%%%%%%%%%%%%%%%%%%%%%%
%%%%%%%%%% Context %%%%%%%%%%%%%%%%%%%%%%%%%%%%%%%%%%%%%%%%%%%%%%%%%%%%%%%%
%%%%%%%%%%%%%%%%%%%%%%%%%%%%%%%%%%%%%%%%%%%%%%%%%%%%%%%%%%%%%%%%%%%%%%%%%%%

%%%%%%%%%% Contexte général %%%%%%%%%%%%%%%%%%%%%%%%%%%%%%%%%%%%%%%%%%%%%%%

En informatique, la compatibilité est la capacité d'un logiciel ou d'un matériel de fonctionner sur les différents systèmes, sans en altérer le fonctionnement.
Dans le domaine du logiciel, cela peut concerner un système d'exploitation, un programme, ou même une librairie.
Le problème de compatibilité le plus connu est celui entre Windows, marque propriétaire, et Linux, solution gratuite et ouverte à tous.


%%%%%%%%%% Context de l'entreprise %%%%%%%%%%%%%%%%%%%%%%%%%%%%%%%%%%%%%%%%

Dans le domaine de la recherche par exemple, il existe des outils scientifiques qui ne sont compatibles uniquement sous Linux.
Lorsque les utilisateurs possèdent un ordinateur ayant comme système d'exploitation Windows, il leur est donc impossible de les utiliser de manière simple.
\\



%%%%%%%%%%%%%%%%%%%%%%%%%%%%%%%%%%%%%%%%%%%%%%%%%%%%%%%%%%%%%%%%%%%%%%%%%%%
%%%%%%%%%% Problème %%%%%%%%%%%%%%%%%%%%%%%%%%%%%%%%%%%%%%%%%%%%%%%%%%%%%%%
%%%%%%%%%%%%%%%%%%%%%%%%%%%%%%%%%%%%%%%%%%%%%%%%%%%%%%%%%%%%%%%%%%%%%%%%%%%

Cependant, il existe plusieurs techniques permettant de contourner ce problème, chacune présentant des avantages et des inconvénients.
\begin{itemize}
	\item La cross-compilation, qui consiste à compiler le programme sur la machine compatible avant d'exporter le programme vers la machine désirée, n'est possible que dans de rares cas.
	\item L'installation d'un autre système d'exploitation, compatible avec l'outil, sur une autre partition du disque-dur, appelé le "multi-boot", demande un redémarrage de l'ordinateur entre chaque échange de système d'exploitation.
	\item Enfin, la virtualisation, quant à elle, consiste à émuler un système d'exploitation à l'intérieur même du système d'exploitation utilisé.
Cette solution est possible dans tous les cas, et permet le meilleur compromis entre temps de mise en œuvre, et puissance de calcul.
\\
\end{itemize}



%%%%%%%%%%%%%%%%%%%%%%%%%%%%%%%%%%%%%%%%%%%%%%%%%%%%%%%%%%%%%%%%%%%%%%%%%%%
%%%%%%%%%% Objectif %%%%%%%%%%%%%%%%%%%%%%%%%%%%%%%%%%%%%%%%%%%%%%%%%%%%%%%
%%%%%%%%%%%%%%%%%%%%%%%%%%%%%%%%%%%%%%%%%%%%%%%%%%%%%%%%%%%%%%%%%%%%%%%%%%%

L'objet de cette étude sera d'étudier et de développer une solution de virtualisation, simple d'utilisation voir transparente pour l'utilisateur, qui permettra d'utiliser des outils à partir d'une interface graphique.
\\



%%%%%%%%%%%%%%%%%%%%%%%%%%%%%%%%%%%%%%%%%%%%%%%%%%%%%%%%%%%%%%%%%%%%%%%%%%%
%%%%%%%%%% Démarche et plan %%%%%%%%%%%%%%%%%%%%%%%%%%%%%%%%%%%%%%%%%%%%%%%
%%%%%%%%%%%%%%%%%%%%%%%%%%%%%%%%%%%%%%%%%%%%%%%%%%%%%%%%%%%%%%%%%%%%%%%%%%%

%%%%%%%%%% Démarche %%%%%%%%%%%%%%%%%%%%%%%%%%%%%%%%%%%%%%%%%%%%%%%%%%%%%%%

La construction d'une première machine virtuelle complète permettra de tester et de comparer les différentes techniques possibles et utilisables dans la solution finale.
Une fois ces techniques définies, elles seront optimisées et intégrées à une seconde machine virtuelle minimale pour obtenir les meilleurs résultats possibles.
Enfin, l'intégration de la solution dans une interface graphique permettra l'utilisation pratique et efficace d'outils, à l'origine non exécutables.


%%%%%%%%%% Plan %%%%%%%%%%%%%%%%%%%%%%%%%%%%%%%%%%%%%%%%%%%%%%%%%%%%%%%%%%%

Tout d'abord, j'analyserai précisément le sujet, en présentation l'entreprise, la virtualisation, ainsi que la solution envisagée.
Dans un deuxième temps je développerai la démarche suivie pour parvenir à la solution finale du projet.
Pour terminer, je présenterai le développement et la solution obtenue.