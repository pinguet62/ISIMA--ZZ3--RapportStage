\cleardoublepage

\chapter{Conception de la solution}

%%%%%%%%%%%%%%%%%%%%%%%%%%%%%%%%%%%%%%%%%%%%%%%%%%%%%%%%%%%%%%%%%%%%%%%%%%%
%%%%%%%%%%%%%%%%%%%%%%%%%%%%%%%%%%%%%%%%%%%%%%%%%%%%%%%%%%%%%%%%%%%%%%%%%%%
%%%%%%%%%%%%%%%%%%%%%%%%%%%%%%%%%%%%%%%%%%%%%%%%%%%%%%%%%%%%%%%%%%%%%%%%%%%
%%%%%%%%%%%%%%%%%%%%%%%%%%%%%%%%%%%%%%%%%%%%%%%%%%%%%%%%%%%%%%%%%%%%%%%%%%%
%%%%%%%%%%%%%%%%%%%%%%%%%%%%%%%%%%%%%%%%%%%%%%%%%%%%%%%%%%%%%%%%%%%%%%%%%%%

\section{L'équipe du projet}

Pour réaliser ce projet, j'ai été intégré dans une petite équipe de trois personnes.

Un chef de projet qui est le relai entre Sopra Group et Limagrain affin d'établir la compréhension du besoin fonctionnel et technique. Sa charge est d'une demi-journée par semaine. De plus, il est chargé de la rédaction des spécifications de l'application. Pour des raisons de délais trop courts et charges limitées, ces spécifications ont été réalisées en parallèle du développement, malgré l'éloignement des bonnes pratiques.

Un architecte a pour objectif de suivre le développeur, conseillant les outils, technologies et méthodes utilisées. Sa charge sur le projet est aussi d'une demi-journée par semaine.

Et enfin moi même, en tant que développeur, en charge du développement de l'application, à plein temps.

%%%%%%%%%%%%%%%%%%%%%%%%%%%%%%%%%%%%%%%%%%%%%%%%%%%%%%%%%%%%%%%%%%%%%%%%%%%
%%%%%%%%%%%%%%%%%%%%%%%%%%%%%%%%%%%%%%%%%%%%%%%%%%%%%%%%%%%%%%%%%%%%%%%%%%%
%%%%%%%%%%%%%%%%%%%%%%%%%%%%%%%%%%%%%%%%%%%%%%%%%%%%%%%%%%%%%%%%%%%%%%%%%%%
%%%%%%%%%%%%%%%%%%%%%%%%%%%%%%%%%%%%%%%%%%%%%%%%%%%%%%%%%%%%%%%%%%%%%%%%%%%
%%%%%%%%%%%%%%%%%%%%%%%%%%%%%%%%%%%%%%%%%%%%%%%%%%%%%%%%%%%%%%%%%%%%%%%%%%%

\section{La relation client}

TODO: détails assistance et différence avec projet (?)

Le projet s'est réalisé en \textit{assistance technique}. Il n'y a aucune date de livré imposée, par contre la charge (nombre de jours d'interventions) totale est fixe.

J'ai travaillé à raison de trois jours chez le client Limagrain et deux jour au siège de Sopra Group, profitant ainsi de la proximité du client pour éclaircir les points ambigus.
\\

La relation directe avec le client m'a permis de définir le besoin du client au cours des différentes étapes du projet.

La première étape a consisté à établir le modèle de données. Pour cela il a fallu étudier la solution existante pour déterminer les informations à mémoriser, leur type, les contraintes, ainsi que les liens entre eux. Cette partie est importante car la structure de l'application est fortement liée a ce modèle.

Deuxièmement, j'ai réalisé une première version de l'interface graphique de l'application. L'objectif est de proposer une première maquette au client qui pourra effectuer des critiques. L'ergonomie de la solution s'est ainsi optimisé au fur et à mesure des démonstrations.

Enfin, une fois l'application fonctionnelle, il était nécessaire d'importer les données existantes dans la nouvelle de données. Comme ces deux bases de données sont différentes, il était nécessaire d'être en contact avec le client pour faire correspondre les schémas.

%%%%%%%%%%%%%%%%%%%%%%%%%%%%%%%%%%%%%%%%%%%%%%%%%%%%%%%%%%%%%%%%%%%%%%%%%%%
%%%%%%%%%%%%%%%%%%%%%%%%%%%%%%%%%%%%%%%%%%%%%%%%%%%%%%%%%%%%%%%%%%%%%%%%%%%
%%%%%%%%%%%%%%%%%%%%%%%%%%%%%%%%%%%%%%%%%%%%%%%%%%%%%%%%%%%%%%%%%%%%%%%%%%%
%%%%%%%%%%%%%%%%%%%%%%%%%%%%%%%%%%%%%%%%%%%%%%%%%%%%%%%%%%%%%%%%%%%%%%%%%%%
%%%%%%%%%%%%%%%%%%%%%%%%%%%%%%%%%%%%%%%%%%%%%%%%%%%%%%%%%%%%%%%%%%%%%%%%%%%

\section{Technologies et architecture}

L'architecture d'une application est importante, car cela définie sa maintenabilité et son l'évolutivité. De plus cela permet la séparation des problèmes diminuant ainsi la complexité.

Notre application est divisée en 3 couches, appelé \textit{3-tiers}, qui est le modèle multi-tiers le plus utilisé. Cela permet de séparer l'accès aux données de la base, la partie métier effectuant les traitements, et l'interface de l'utilisateur. La figure \ref{architecture_3_tiers} représente la séparation de ces différentes couches. Dans cette partie je présenterai ces différentes couches, en détaillant leurs fonctions, leurs interactions et les technologies utilisées.
%\begin{figure}[!h]
%	\center
%	\includegraphics[width=2cm]{img/architecture_3_tiers.png}
%	\caption{Architecture 3 tiers}
%	\label{architecture_3_tiers}
%\end{figure}
~~\\

Le groupe Limagrain travaille dans un environnement Micosoft et utilise ainsi leurs technologies et logiciels, comme par exemple : Windows pour le système d'exploitation, Internet Explorer comme navigateur internet, Outlook comme messagerie, \ldots

Pour développer cette solution nous nous sommes donc tourné vers les technologies Microsoft, facilitant ainsi la compatibilité et l'intégration des composants.

%%%%%%%%%%%%%%%%%%%%%%%%%%%%%%%%%%%%%%%%%%%%%%%%%%%%%%%%%%%%%%%%%%%%%%%%%%%
%%%%%%%%%%%%%%%%%%%%%%%%%%%%%%%%%%%%%%%%%%%%%%%%%%%%%%%%%%%%%%%%%%%%%%%%%%%
%%%%%%%%%%%%%%%%%%%%%%%%%%%%%%%%%%%%%%%%%%%%%%%%%%%%%%%%%%%%%%%%%%%%%%%%%%%

\subsection{Socle}

% TODO: socle = framework ?
La solution développée se base sur un socle existant, développé dans le cadre d'un autre projet. Ceci nous a permis un gain de temps important car une partie importante du projet n'a du être développé à nouveau. En contre partie, les différentes technologies utilisées, qui seront présentées dans cette partie, nous ont été imposées.

%%%%%%%%%%%%%%%%%%%%%%%%%%%%%%%%%%%%%%%%%%%%%%%%%%%%%%%%%%%%%%%%%%%%%%%%%%%

\subsubsection{Gestion des utilisateurs}

Le socle permet une gestion des utilisateurs de l'application. Il est possible de les créer, modifier ou supprimer, pour restreindre l'accès aux personnes autorisées. De plus, il est possible de connecter l'application à un LDAP.
\\

Le \textit{LDAP}, pour Lightweight Directory Access Protocol, est un protocole standard de gestion d'utilisateurs. Son objectif est de centraliser les informations des utilisateurs (nom, identifiant, mot de passe, \ldots) dans un annuaire.

Le principal avantage de cette solution est la mise en commun des comptes utilisateurs, ce qui permet l'utilisateur d'un seul et même compte pour tous les services connectés : Windows, la boite mail Outlook, cette application, \ldots. La figure \ref{LDAP} représente l'interaction des applications avec un LDAP.
%\begin{figure}[!h]
%	\center
%	\includegraphics[width=2cm]{img/LDAP.png}
%	\caption{Schéma d'un LDAP}
%	\label{LDAP}
%\end{figure}

%%%%%%%%%%%%%%%%%%%%%%%%%%%%%%%%%%%%%%%%%%%%%%%%%%%%%%%%%%%%%%%%%%%%%%%%%%%

\subsubsection{Gestion des droits}

Il est aussi possible de gérer des droits dans l'application grâce à ce socle. L'objectif est de contrôler les accès et les actions aux utilisateurs ayant le droit, pour protéger les informations confidentielles.
\\

Chaque écran de l'application peut avoir leur accès ou certaines actions restreintes en fonction d'un \textit{droit}. Ceux-ci peuvent prendre trois valeur : 
\begin{itemize}
	\item "aucun accès", par défaut, qui interdit tout accès à l'écran ;
	\item "lecture" n'autorise que l'accès à l'écran, avec la possibilité d'effectuer des recherches ou d'afficher les détails ;
	\item "lecture et écriture" autorise toutes les actions possibles.
\end{itemize}

Un \textit{profil} est un ensemble de droits, qui permet de définir un périmètre d'action. Par exemple un profil "administrateur" aura tous les droits dans l'application, un profil "manager" n'aura que des droits de lecture, ou encore un profil "ressource humaine" possèdera les droits liés aux candidats et contrats, \ldots

Chaque utilisateur possède un ou plusieurs profils, ce qui permet de définir l'ensemble de ses droits. L'utilisation de profils plutôt que de directement de droit permet un gain de temps, car la personne administrant les utilisateurs n'aura pas à affecter les nombreux droits aux nouveaux utilisateurs, et la modification d'un profil permet d'impacter l'ensemble des utilisateurs associés.
\\

La figure \ref{utilisateur_profils_droits} représente le schéma UML des utilisateurs, profils et droits dans l'application.
%\begin{figure}[!h]
%	\center
%	\includegraphics[width=2cm]{img/utilisateur_profils_droits.png}
%	\caption{Utilisation des profils et droits des utilisateurs}
%	\label{utilisateur_profils_droits}
%\end{figure}

%%%%%%%%%%%%%%%%%%%%%%%%%%%%%%%%%%%%%%%%%%%%%%%%%%%%%%%%%%%%%%%%%%%%%%%%%%%
%%%%%%%%%%%%%%%%%%%%%%%%%%%%%%%%%%%%%%%%%%%%%%%%%%%%%%%%%%%%%%%%%%%%%%%%%%%
%%%%%%%%%%%%%%%%%%%%%%%%%%%%%%%%%%%%%%%%%%%%%%%%%%%%%%%%%%%%%%%%%%%%%%%%%%%

\subsection{Langage de programmation}

%%%%%%%%%%%%%%%%%%%%%%%%%%%%%%%%%%%%%%%%%%%%%%%%%%%%%%%%%%%%%%%%%%%%%%%%%%%

\subsubsection{Microsoft .NET Framework}

\textit{.NET Framework} est une plateforme application proposée par Microsoft. Cette technologie est comparable et directement concurrente de Java d'Oracle.

Ce framework est constitués de nombreux composants, disposés en couches au fur et à mesure de ses versions, schématisés sur la figure \ref{.NET_Framework} :
\begin{enumerate}
	\item Le moteur d'exécution appelé Common Language Runtime (CLR) permet de compiler le code course en un langage intermédiaire appelé Microsoft Intermediate Language (MSIL). Ce code est ensuite compilé à la volée lors de la première exécution grâce au compilateur "Just In Time" (JIT) ;
	\item Une bibliothèque de classes, offrant des fonctionnalités de base pour les différentes applications ;
	\item Plusieurs couches supplémentaires proposant des outils de développement d'interfaces graphique (WinForms), d'accès aux données (ADO.NET (Entity Framework)), \ldots
%	\item Une couche de trois composants : WinForms (ou Windows Forms) pour le développement d'interfaces graphiques, ASP.NET permettant la création de sites web dynamiques, et ADO.NET pour l'accès aux bases de données ;
%	\item La version 3.0 apporte Windows Presentation Foundation (WPF) permettant le développement d'applications graphiques vectorielles basées sur le XML, Windows Communication Foundation (WCF) pour la communication, Windows Workflow Foundation (WF) une technologie de gestion des workflow, et enfin Windows CardSpace pour la gestion d'identités ;
\end{enumerate}

%%%%%%%%%%%%%%%%%%%%%%%%%%%%%%%%%%%%%%%%%%%%%%%%%%%%%%%%%%%%%%%%%%%%%%%%%%%

\subsubsection{VB.NET}

\textit{Visual Basic .NET} est un langage de programmation développé par Microsoft. Il s'agit d'une évolution majeure de Visual Basic 6, introduisant l'aspect orienté objet. De plus, le code est compilé dans un langage intermédiaire, appelé Common Intermediate Language (CIL), au même titre que les autres langages fonctionnant sur la machine virtuelle .Net. Ce langage est très proche du C\#, à la syntaxe près.

Le projet a été développé dans ce langage, aussi bien la couche métier que dans la couche présentation.

%%%%%%%%%%%%%%%%%%%%%%%%%%%%%%%%%%%%%%%%%%%%%%%%%%%%%%%%%%%%%%%%%%%%%%%%%%%
%%%%%%%%%%%%%%%%%%%%%%%%%%%%%%%%%%%%%%%%%%%%%%%%%%%%%%%%%%%%%%%%%%%%%%%%%%%
%%%%%%%%%%%%%%%%%%%%%%%%%%%%%%%%%%%%%%%%%%%%%%%%%%%%%%%%%%%%%%%%%%%%%%%%%%%

\subsection{Base de données}

Une \textit{base de données} permet de stocker un grand nombre d'informations ayant des natures différentes. Ces informations sont stockées dans des tables, où les clés permettent de définir les liens de dépendant entre les informations.

%%%%%%%%%%%%%%%%%%%%%%%%%%%%%%%%%%%%%%%%%%%%%%%%%%%%%%%%%%%%%%%%%%%%%%%%%%%

\subsubsection{PowerAMC}

\textit{PowerAMC}, version francophone de PowerDesigner, est un logiciel de modélisation de base de données produit par la société Sybase. Il permet d'établir facilement un modèle de données à partir de son interface graphique visible sur la figure \ref{PowerAMC} TODO: ici
%\begin{figure}[!h]
%	\center
%	\includegraphics[width=2cm]{img/PowerAMC.png}
%	\caption{Interface de PowerAMC}
%	\label{PowerAMC}
%\end{figure}

%%%%%%%%%%%%%%%%%%%%%%%%%%%%%%%%%%%%%%%%%%%%%%%%%%%%%%%%%%%%%%%%%%%%%%%%%%%

\subsubsection{SQL Server 2008}

Il existe de nombreux système de gestion de base de données : Oracle, MySQL, PostgreSQL, \ldots Nous avons utilisé \textit{Microsoft SQL Server}, dans sa version 2008 R2, par demande du client.

TODO: licence : ce choix car licence déjà achetée pour d'autres ?

%%%%%%%%%%%%%%%%%%%%%%%%%%%%%%%%%%%%%%%%%%%%%%%%%%%%%%%%%%%%%%%%%%%%%%%%%%%

\subsubsection{Mapping de la base de données}

Le \textit{modèle relationnel} est utilisé dans les systèmes de gestion de base de données (SGBD) pour rassembler un ensemble d'informations. Les données (clés) sont dupliquées entre les tables et l'accès aux relations s'effectue ensuite grâce à des jointures entre les tables.

Le \textit{modèle objet}, quant à lui, est utilisé dans la programmation orientée objet. Les données sont modélisées sous la formes d'objets, entités complexes ayant des comportements et des relations entre elles.
\\

Le \textit{mapping objet-relationnel} consiste à interfacer le modèle relationnel d'une base de données avec le modèle orienté objet d'un programme informatique. Généralement, une classe modélisera une table, et attribut d'objet modélisera un champ d'une table, avec un type similaire (par exemple \lstinline{String} pour \lstinline{varchar}).
\\

Cette opération peut être faite à l'aide d'un framework (TODO: glossaire), permettent de s'abstraire de la base de données, automatisant et réduisant ainsi la duplication de code. L'objectif est de faciliter le développement, augmenter la maintenabilité du programme, ou encore s'abstenir du type de base de données.

Microsoft propose plusieurs framework, et c'est \textit{Entity Framework} qui a été utilisé. Il est intégré à Visual Studio, ce qui permet une génération et un paramétrage facilité. De plus, il permet l'interaction avec  LINQ (Language-Integrated Query), extension du langage permettant de faire des requêtes sur des ensembles de données s'abstrayant du type.

%%%%%%%%%%%%%%%%%%%%%%%%%%%%%%%%%%%%%%%%%%%%%%%%%%%%%%%%%%%%%%%%%%%%%%%%%%%
%%%%%%%%%%%%%%%%%%%%%%%%%%%%%%%%%%%%%%%%%%%%%%%%%%%%%%%%%%%%%%%%%%%%%%%%%%%
%%%%%%%%%%%%%%%%%%%%%%%%%%%%%%%%%%%%%%%%%%%%%%%%%%%%%%%%%%%%%%%%%%%%%%%%%%%

\subsection{Les services}

%%%%%%%%%%%%%%%%%%%%%%%%%%%%%%%%%%%%%%%%%%%%%%%%%%%%%%%%%%%%%%%%%%%%%%%%%%%

\subsubsection{Fonction}

La couche de \textit{service} est la partie métier de l'application. Cela permet de séparer le code source lié à l'affichage destiné à l'utilisateur, du code métier effectuant des manipulations dans la base de données.

%%%%%%%%%%%%%%%%%%%%%%%%%%%%%%%%%%%%%%%%%%%%%%%%%%%%%%%%%%%%%%%%%%%%%%%%%%%

\subsubsection{Client-Serveur}

\Jparagraph{Web service}

Un \textit{service web} (ou \textit{web service}) est un programme informatique permettant la communication et l'échange d'informations entre des systèmes hétérogènes et distribués (local, réseau, internet, \ldots), exposant ainsi des fonctionnalités.

L'échange d'informations entre le client et le serveur se fait par sérialisation, consistant à coder les informations contenues en mémoire. Cela peut se faire sous le format texte (XML, JSON, \ldots) ou au format binaire.


\Jparagraph{Windows Communication Foundation}

\textit{Windows Communication Foundation}, couramment appelé sous ses initiales WCF, est la couche de communication de .NET Framework. Cette technologie respecte les normes standards des services web, ce qui lui permet d'appeler ou d'être appelé par des technologies différentes (Java, Python, \ldots).


\Jparagraph{WCF RIA Services}

Il est souvent nécessaire de posséder une logique applicative à la fois du coté serveur que du coté client. C'est le cas par exemple lorsque l'on souhaite vérifier la validité des données avant de les insérer en base de données : soit on effectue la même vérification du coté client et serveur, ce qui impose de dupliquer le code dans les deux couches, soit on n'effectue la vérification coté serveur, ce qui impose une communication inutile entre client et serveur.

Pour éviter ce problème, Microsoft propose le framework \textit{WCF RIA Services}. Cet outil génère du code, contenu du coté serveur, dans la couche client, lors de la compilation.
\\


La figure \ref{WCF_RIA_Services} représente les échanges entre la partie cliente de l'application et la partie serveur.
%\begin{figure}[!h]
%	\center
%	\includegraphics[width=2cm]{img/WCF_RIA_Services.png}
%	\caption{Fonctionnement de l'application client-serveur}
%	\label{WCF_RIA_Services}
%\end{figure}

%%%%%%%%%%%%%%%%%%%%%%%%%%%%%%%%%%%%%%%%%%%%%%%%%%%%%%%%%%%%%%%%%%%%%%%%%%%
%%%%%%%%%%%%%%%%%%%%%%%%%%%%%%%%%%%%%%%%%%%%%%%%%%%%%%%%%%%%%%%%%%%%%%%%%%%
%%%%%%%%%%%%%%%%%%%%%%%%%%%%%%%%%%%%%%%%%%%%%%%%%%%%%%%%%%%%%%%%%%%%%%%%%%%

\subsection{Interface utilisateur}

%%%%%%%%%%%%%%%%%%%%%%%%%%%%%%%%%%%%%%%%%%%%%%%%%%%%%%%%%%%%%%%%%%%%%%%%%%%

\subsubsection{Silverlight}

Microsoft \textit{Silverlight} est un plugin pour navigateur web. Il permet le développement d'applications riches et de pousser plus loin l'expérience utilisateur du web 2.0, au même titre qu'Adobe Flash dont il se veut une alternative. Initialement prévu pour des applications web dans un navigateur, les programmes peuvent être téléchargées pour être utilisés directement sur l'ordinateur ("out of browser"), et permettent aussi le développement d'applications pour Windows Phone 7.

Cette technologie nécessite l'installation du plugin, qui est un sous-ensemble de Microsoft .NET Framework. Les applications sont ainsi cross-browser (Internet Explorer, Firefox, Chrome, \ldots) et cross-platform (Windows, OS-X et Linux, via le projet open-source Moonlight). De plus les applications fonctionnent dans une "sandbox" ("bac a sable") ce qui permet de garantir une sécurité accrue pour l'utilisateur et le serveur.

Une des spécifications de Silverlight est l'impossibilité d'appeler des services web de manière synchrone, c'est à dire que la fonction appelante attend la réponse. En effet, lors d'un appel l'exécution continue et une événement est émis lors de la réception de la réponse. Cette solution permet de ne pas bloquer l'interface de l'utilisateur, qui pourrait penser à un plantage de l'application. Mais l'inconvénient est l'augmentation de la complexité de programmation car il est nécessaire de contrôler plusieurs fils d'exécution en parallèle.

~~\\-----------------------------
Reporting services
Parler des versions ?
SignalR
Log4Net